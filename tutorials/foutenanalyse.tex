\documentclass[10pt,a4paper]{article}
\usepackage[latin1]{inputenc}
\usepackage[dutch]{babel}
\usepackage{amsmath}
\usepackage{amsfonts}
\usepackage{amssymb}

\author{Tom Sydney Kerckhove}
\title{Tutorial: Foutenanalyse}
\date{Started: 27 februari 2014\\ Compiled: \today}

\begin{document}

\maketitle
\tableofcontents

\section{Voorkennis}
\begin{itemize}
\item Calculus
\end{itemize}

\pagebreak

\section{Theorie}
\subsection{Absolute en relatieve fout}
De absolute fout is gedefinieerd als het verschil tussen de berekende waarde en de echte waarde.
\[
err_{abs} = \Delta y = \overline{y} - y
\]
De relatieve fout is de absolute fout relatief ten opzichte van het echte resultaat.
\[
err_{rel} = \frac{\Delta y}{y} = \frac{\overline{y} - y}{y}
\]

\subsection{$\overline{y}$ exact berekenen}
Bij elke bewerking maakt de machine een afrondingsfout, begrensd door de machine precisie.
\[
\epsilon_i \le \epsilon_{mach}
\]
Zij $\circ$ een elementaire bewerking, dan geldt het volgende.
\[
fl(a\circ b) = (a\circ b)(1+\epsilon_{\circ})
\]

\subsection{Maclaurin reeks}
Een Maclaurin reeks rond is een reeks die een functie benadert in de buurt van $0$. Zij $f(x)$ de te benaderen functie met $x$ een variabele.
\[
f(x) = \sum_{i=0}^\infty\frac{f^{(i)}(0)}{i!}x^i = f(0) + f'(0)x + \frac{1}{2}f''(0)x^2 + ...
\]
Of nog algemener, met $n$ variabelen $x_1,...,x_n$.
\[
f(x_1,...,x_n) = \sum_{i=0}^\infty\sum_{j=0}^{n}\frac{x_j^i}{i!}\frac{\delta f}{\delta x_j}(0,...,0)
\]

\subsection{$\overline{y}$ benaderen met een Maclaurin reeks}
Eens we $\overline{y}$ exact berekent hebben kunnen we $\overline{y}$ (vrij goed) benaderen zodat de vergelijking eenvoudiger wordt. Zij $n$ het aantal berekeningen om $\overline{y}$ te bekomen. $\overline{y}$ is dus een functie in $\epsilon_1,...,\epsilon_n$ met $x$ als parameter. We gebruiken een Maclaurin reeks om $\overline{y}$ te benaderen.
\[
\overline{y} = y + \sum_{i=1}^{n}\frac{\delta\overline{y}}{\delta\epsilon_i}(0,...,0)\epsilon_i
\]
We laten de hogere orde termen weg, omdat een term met $\epsilon_{i}^2$ een verwaarloosbare invloed heeft. 

\section{Algoritme}
Gegeven een getal $y$ dat we berekenen met een machine zodat we $\overline{y}$ bekomen. We berekenen nu de absolute en relatieve fout op deze berekening.
\begin{enumerate}
\item Bereken $\overline{y}$ exact.
\begin{itemize}
\item Duid de bewerkingen aan.
\item Zet rond elke bewerking $fl( ... )$.
\item Voeg aan het resultaat van elke bewerking $(1+\epsilon_i)$ toe.
\end{itemize}

\item Bepaal de waarde van de parti\"ele afgeleide van $\overline{y}$ naar elke $\epsilon_i$ en evalueer deze in $(0,...,0)$. Hint: de variabelen naar dewelke u niet afleidt kan u meteen invullen.

\item Benader $\overline{y}$ met een Maclaurin reeks.

\item Benader $err_{abs}$ en $err_{rel}$.
\end{enumerate}

\section{Voorbeeld}
\[
y = \frac{1-\cos(x)}{x^2}
\]
\begin{enumerate}
\item
\[
\overline{y} = fl\left(\frac{fl(1-fl(\cos(x)))}{fl(x^2)}\right)
\]
\[
\overline{y} = \frac{(1-(\cos(x))(1+\epsilon_1))(1+\epsilon_2)}{(x^2)(1+\epsilon_3)}(1+\epsilon_4)
\]

\item
\[
\frac{\delta \overline{y}}{\delta \epsilon_1}(\epsilon_1,0,0,0)
= \frac{\delta}{\delta \epsilon_1}\frac{1-(\cos(x))(1+\epsilon_1)}{x^2} 
= -\frac{\cos(x)}{x^2}
\]
\[
\frac{\delta \overline{y}}{\delta \epsilon_2}(0,\epsilon_2,0,0)
= \frac{\delta}{\delta \epsilon_2}\frac{(1-\cos(x))(1+\epsilon_2)}{x^2}
= \frac{(1-\cos(x))}{x^2}
\]
\[
\frac{\delta \overline{y}}{\delta \epsilon_3}(0,0,\epsilon_3,0)
= \frac{\delta}{\delta \epsilon_3}\frac{1-\cos(x)}{(x^2)(1+\epsilon_3)} = \frac{1-\cos(x)}{(x^2)}\frac{-1}{(1+\epsilon_3)^2}
\]
\[
\frac{\delta \overline{y}}{\delta \epsilon_3}(0,0,0,0) = -\frac{1-\cos(x)}{x^2}
\]
\[
\frac{\delta \overline{y}}{\delta \epsilon_4}(0,0,0,\epsilon_4)
= \frac{\delta}{\delta \epsilon_4}\frac{(1-\cos(x))(1+\epsilon_4)}{x^2}
= \frac{1 - \cos(x)}{x^2}
\]

\item
\[
\overline{y} \approx y -\frac{\cos(x)}{x^2}\epsilon_1
+ \frac{(1-\cos(x))}{x^2}\epsilon_2
- \frac{1-\cos(x)}{x^2}\epsilon_3
+ \frac{1 - \cos(x)}{x^2}\epsilon_4
\]
\[
\overline{y} \approx y - \frac{\cos(x)}{x^2}\epsilon_1 + y\epsilon_2 - y\epsilon_3 + y\epsilon_4
\]

\item
\[
\overline{y}-y \approx - \frac{\cos(x)}{x^2}\epsilon_1 + y\epsilon_2 - y\epsilon_3 + y\epsilon_4
\]
\[
\frac{\overline{y}-y}{y} \approx - \frac{\cos(x)}{x(1-\cos(x))}\epsilon_1 + \epsilon_2 - \epsilon_3 + \epsilon_4
\]

\end{enumerate}

\end{document}