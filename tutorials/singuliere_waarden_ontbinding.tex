\documentclass[10pt,a4paper]{article}

\usepackage[latin1]{inputenc}
\usepackage[dutch]{babel}
\usepackage{amsmath}
\usepackage{amsfonts}
\usepackage{amssymb}
\usepackage{amsthm}

\author{Tom Sydney Kerckhove}
\title{Tutorial: Singuliere Waarden Ontbinding}
\date{Started: 5 maart 2014\\Compiled: \today}

\begin{document}

\maketitle
\tableofcontents
\pagebreak

\section{Inleiding}

\subsection{Achtergrond}
\subsubsection{Getransponeerde matrix}
De getransponeerde matrix $A^T$ van een matrix $A$ wordt bekomen door de matrix te spiegelen ten opzichte van de hoofd diagonaal.
Met andere woorden: element $a_{ij}$ uit $A$ is element $a_{ji}$ uit $A^T$.\\\\
De toegevoegd getransponeerde matrix $A^*$ is het elementsgewijze complex toegevoegde van $A^T$.

\subsubsection{Unitaire Matrix}
Een orthogonale matrix is een re\"ele matrix $A$ waarvoor $A^TA=AA^T=\mathbb{I}$ geldt. $A^T$ is dus de inverse van $A$.
Een unitaire matrix is een complexe matrix $U$ waarvoor $U^*U=UU^*=\mathbb{I}$ geldt. $U^*$ is dus de inverse van $U$.

\section{Definitie}
De singuliere waarden ontbinding van een $m\times n$ matrix $M$ is een decompositie van de volgende vorm.
\[
M = U \Sigma V^*
\]
\begin{itemize}
\item $U$ is een unitaire $m\times m$ matrix.
\item $\Sigma$ is een $m\times n$ pseudo-diagonaal matrix met positieve re\"ele getallen op de diagonaal.
\item $V$ is een unitaire $n\times n$ matrix.
\end{itemize}
De singuliere waarden van $M$ staan op de diagonaal van $M$.\\\\
\[
M\vec{v} = \sigma\vec{u} \text{ en } M^*\vec{u} = \sigma\vec{v}
\]
$\vec{u}$ en $\vec{v}$ worden respectievelijk de linker en rechter-singuliere vectoren genoemd.

\section{Ontbinding}
\subsection{Theorie}
TODO
\subsection{Algoritme}
TODO
\subsection{Voorbeeld}
TODO

\end{document}