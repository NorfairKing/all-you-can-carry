\documentclass[12pt,a4paper]{article}
\usepackage[latin1]{inputenc}
\usepackage[dutch]{babel}
\usepackage{amsmath}
\usepackage{amsfonts}
\usepackage{amssymb}
\usepackage{amsthm}
\usepackage{enumerate}
\usepackage{todonotes}
\usepackage[left=2cm,right=2cm,top=2cm,bottom=2cm]{geometry}

\author{Tom Sydney Kerckhove}
\title{Oefeningen Numerieke Wiskunde:\\ Oefenzitting 5}
\date{11 maart 2014}

\begin{document}
\maketitle

\section{Probleem 1}
Zie de matlab scripts.
Voor de eerste formule bekomen we maximaal $8$ juiste beduidende cijfers, voor de tweede bekomen we er maximaal $11$. We plotten de relatieve fout ten opzichte van $h$ in een loglog-plot en zien duidelijk twee componenten in de fout. Eenderzijds links de afrondingsfouten en anderzijds rechts de benaderingsfouten.

\section{Probleem 2}
De ordes van de benaderingen zijn respectievelijk $1$ en $2$.
Ontwikkel $f(x+h)$ en $f(x-h)$
\[
f(x+h) = f(x) + f'(x)h + f''(x)h^2 + \cdots
\]
\[
f(x-h) = f(x) - f'(x)h + f''(x)h^2 + \cdots
\]
Vul nu in:\\
Voor $y_1$:
\[
y_1 = \frac{f(x+h) - f(x)}{h} = \frac{f(x) + f'(x)h + f''(x)h^2 + \cdots - f(x)}{h} = f'(x) + f''(x)h + \cdots
\]
\[
= f'(x) + O(h) 
\]
Voor $y_2$:
\[
y_2 = \frac{f(x+h) - f(x-h)}{2h} 
\]
\[= \frac{f(x) + f'(x)h + f''(x)h^2 + f'''(x)h^3 + \cdots - f(x) + f'(x)h - f''(x)h^2 + + f'''(x)h^3 + \cdots }{2h}
\]
\[
f'(x) + f'''(x)h^2 + \cdots = f'(x) + O(h^2)
\]


\section{Probleem 3}
Het bewijs is volledig analoog.

\section{Probleem 4}
Zie matlab: er is geen verschil.

\end{document}