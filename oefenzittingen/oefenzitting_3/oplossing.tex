\documentclass[12pt,a4paper]{article}
\usepackage[latin1]{inputenc}
\usepackage[dutch]{babel}
\usepackage{amsmath}
\usepackage{amsfonts}
\usepackage{amssymb}
\usepackage{amsthm}
\usepackage{enumerate}

\author{Tom Sydney Kerckhove}
\title{Oefeningen Numerieke Wiskunde:\\ Oefenzitting 3}
\date{26 februari 2014}

\begin{document}
\maketitle

\section{Probleem 1}
\[
b = 2 \text{ en } p = 53
\]
\[
eps = 2.2204e-16\ \text{ en } \epsilon_{mach} = 1.1102e-16
\]
Voorbeeld van een afrondingsfout:
\[
(1+0.70\cdot eps) -1 = 2.2204e-16
\]

\section{Probleem 2}
\begin{itemize}
\item Correct:\\
$0.125$, $0.25$, $0.5$, $1$, $2$, $8$, $1$, $10$ ,$100$, $1000$.
\item Incorrect:\\
$0.001$, $0.01$, $0.1$
\end{itemize}
$0.1$ wordt voorgesteld als
{\scriptsize \begin{verbatim}
+1.1001100110011001100110011001100110011001100110011010 2^01111111011 (2^-4)
\end{verbatim}}
De fout is dus $fl(0.1)-0.1$.
{\scriptsize \begin{verbatim}
-0.000000000000000000000000000000000000000000000000000100110011...  2^01111111011 (2^-4)
\end{verbatim}}
\[
\Delta(0.1) = (1.00110011..)_{2} \cdot 2^{-56}
\]

\section{Probleem 3}
\begin{enumerate}[(a)]
\item
De waarde van $k$ verandert niet meer na een bepaalde iteratie omdat de component die nog toegevoegd wordt kleiner is dan voorstelbaar.

\item
Met matlab, zie $abs\_err$ en $rel\_err$.

\item
LogLog of Semilogy geeft het mooiste resultaat.

\item
Zie matlab voor illustratie.

\end{enumerate}

\section{Probleem 4}
Voor grote $n$ delen we door een voorgestelde nul in de machine.
De benadering convergeert beter, maar is ongeveer even goed.
\section{Probleem 5}
\begin{enumerate}[(a)]
\item Voor grote $x$ duurt het langer voor $x^k$ kleiner wordt dan $k!$.
\item Zie matlab voor illustratie.
\item Ja, de relatieve fout is van de grootteorde $10^{-15}$.
\end{enumerate}

\section{Probleem 6}
\begin{enumerate}[(a)]
\item Zie matlab voor illustratie.
\item Nee, de relatieve fout is van de grootteorde $10^{-10}$.
\item TODO
\end{enumerate}


\section{Probleem 7}
\begin{enumerate}[(a)]
\item
\[
\lim_{x \rightarrow 0} \frac{1-\cos(x)}{x^2}
= \lim_{x \rightarrow 0} \frac{\sin(x)}{2x0}
= \lim_{x \rightarrow 0} \frac{\cos(x)}{2}
= \frac{1}{2}
\]

\item
Vanaf een bepaalde waarde wordt de benadering terug slechter, en daarna gaat alles naar de maan.

\item
LogLog werkt het best, omdat zowel $x$ als $abs\_err$ een logaritmisch verloop hebben.

\item
De absolute fout is:
\[
f(x)-\frac{1}{2} = -\frac{1}{2} + frac{1}{2} - \frac{x^4}{4!} + \frac{x^6}{6!} -... = \sim\frac{x^4}{4!}
\]

\item
TODO

\end{enumerate}

\section{Probleem 8}
\begin{enumerate}
\item
{\scriptsize \begin{verbatim}
+0.00011001100110011001100
\end{verbatim}}

\item

\item

\end{enumerate}


\end{document}