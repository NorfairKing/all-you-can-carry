\documentclass[12pt,a4paper]{article}
\usepackage[latin1]{inputenc}
\usepackage[dutch]{babel}
\usepackage{amsmath}
\usepackage{amsfonts}
\usepackage{amssymb}
\usepackage{todonotes}
\usepackage{amsthm}
\usepackage{enumerate}
\usepackage[left=2cm,right=2cm,top=2cm,bottom=2cm]{geometry}

\author{Tom Sydney Kerckhove
		\\ Verbeteringen door Ward Schodts}
\title{Oefeningen Numerieke Wiskunde:\\ Oefenzitting 3}
\date{26 februari 2014}

\begin{document}
\maketitle
\section{Bewegende kommavoorstelling}
\subsection{Probleem 1}
\[
b = 2 \text{ en } p = 53
\]
\[
eps = 2.2204e-16\ \text{ en } \epsilon_{mach} = 1.1102e-16
\]
Voorbeeld van een afrondingsfout:
\[
(1+0.70\cdot eps) -1 = 2.2204e-16
\]
0.70 voor eps is net groot genoeg om eps naar boven laten af te ronden, vandaar dat eps wordt teruggegeven
\subsection{Probleem 2}
\begin{itemize}
\item Correct:\\
$0.125$, $0.25$, $0.5$, $1$, $2$, $8$, $1$, $10$ ,$100$, $1000$.
\item Incorrect:\\
$0.001$, $0.01$, $0.1$
\end{itemize}
$0.1$ wordt voorgesteld als
{\scriptsize \begin{verbatim}
+1.1001100110011001100110011001100110011001100110011010 2^01111111011 (2^-4)
\end{verbatim}}
De fout is dus $fl(0.1)-0.1$.
{\scriptsize \begin{verbatim}
-0.000000000000000000000000000000000000000000000000000100110011...  2^01111111011 (2^-4)
\end{verbatim}}
\[
\Delta(0.1) = (1.00110011..)_{2} \cdot 2^{-56}
\]
\section{Taylorreeks van $e^x$}
\subsection{Probleem 3}
\begin{enumerate}[(a)]
\item
De waarde van $k$ verandert niet meer na een bepaalde iteratie omdat de component die nog toegevoegd wordt kleiner is dan voorstelbaar.

\item
Met matlab, zie $abs\_err$ en $rel\_err$.

\item
LogLog of Semilogy geeft het mooiste resultaat.

\item
Zie matlab voor illustratie.

\end{enumerate}

\subsection{Probleem 4}
Voor grote $n$ delen we door een voorgestelde nul in de machine.
De benadering convergeert beter, maar is ongeveer even goed.
\subsection{Probleem 5}
\begin{enumerate}[(a)]
\item Voor grote $x$ duurt het langer voor $x^k$ kleiner wordt dan $k!$.
\item Zie matlab voor illustratie.
\item Ja, de relatieve fout is van de grootteorde $10^{-15}$.
\item De absolute fout is van grootteorde $10^{-7}$. Aangezien een de bovengrens voor de som van $n$ getallen deels gegeven wordt door: $\epsilon_{mach} \cdot \sum_i^n a_i$. Bij deze waarden is de grootste waarde van t al reeds $10^8$. We weten ook dat de machine precisie gelijk is aan $1.110223024625157e^{-16}$. Dus we zien dat de fout groter wordt als de bovengrens want $10^{-16} \cdot 10^8 = 10^{-8} < 10^{-7}$
\end{enumerate}

\subsection{Probleem 6}
\begin{enumerate}[(a)]
\item Zie matlab voor illustratie.
\item Nee, de relatieve fout is van de grootteorde $10^{1}$.
\item Ja.
\[
\overline{S}-S \approx \sum_{k=2}^n\epsilon_k\sum_{i=1}^ka_i
\]
\[
= \sum_{k=2}^n\epsilon_k\sum_{i=1}^k \frac{x^n}{n!}
\]
Zoals we zien is de absolute fout in het begin heel groot omdat het even duur voordat $x^n$ kleiner wordt dan $n!$.
\end{enumerate}


\section{Probleem 7}
\begin{enumerate}[(a)]
\item
\[
\lim_{x \rightarrow 0} \frac{1-\cos(x)}{x^2}
= \lim_{x \rightarrow 0} \frac{\sin(x)}{2x0}
= \lim_{x \rightarrow 0} \frac{\cos(x)}{2}
= \frac{1}{2}
\]

\item
Vanaf een bepaalde waarde wordt de benadering terug slechter, en daarna gaat alles naar de maan.

\item
LogLog werkt het best, omdat zowel $x$ als $abs\_err$ een logaritmisch verloop hebben.

\item
De absolute fout is:
\[
f(x)-\frac{1}{2} = -\frac{1}{2} + \frac{1}{2} - \frac{x^4}{4!} + \frac{x^6}{6!} -... = \sim\frac{x^4}{4!}
\]

\item
Vervang als volgt:
\[
\cos(x) = 1 - \frac{x^2}{2!} + \frac{x^4}{4!} - \frac{x^6}{6!} + \cdots
\]

\[
\delta y \approx y \left( - \frac{1 - \frac{x^2}{2!} + \frac{x^4}{4!} - \frac{x^6}{6!} + \cdots}{\frac{x^2}{2!} - \frac{x^4}{4!} + \frac{x^6}{6!} - \cdots}\epsilon_1 + \epsilon_2 - \epsilon_3 + \epsilon_4 \right)  = \sim \frac{1}{x^2}\epsilon_1
\]

\end{enumerate}

\section{Probleem 8}
\begin{enumerate}
\item $0.1$ wordt naar het volgende getal afgekapt:
{\scriptsize \begin{verbatim}
+0.00011001100110011001100
\end{verbatim}}
\[
= 0.09999990463256835937
\]
De absolute fout is dus de volgende uitdrukking
{\scriptsize \begin{verbatim}
+0.00000000000000000000000110011001100110011...
=
9.536743164617612e-8
\end{verbatim}}
De relatieve fout is dus de volgende.
{\scriptsize \begin{verbatim}
9.536743164617612e-7
\end{verbatim}}

\item Vermenigvuldigen we de relatieve fout met 100 uur ($60*60*100$), dan krijgen we de fout op de berekende tijd.
\[
0.343
\]

\item
\[
0.343 * 1676 = 574.868m
\]
\end{enumerate}

\section{Probleem 9}
\begin{enumerate}[(a)]
\item
\begin{enumerate}
\item Wortels\\
\[
y = \sqrt[2^{40}]{x}
\]
\begin{enumerate}
\item
\[
\overline{y} = fl\left(\sqrt{...fl\left(\sqrt{fl(\sqrt{x})}\right)...}\right)
\]
\[
\overline{y}
= (1+\epsilon_{40})\sqrt{...(1+\epsilon_2)\sqrt{(1+\epsilon_1)\sqrt{x}}}
\]
\[
= (1+\epsilon_{40})(1+\epsilon_{39})^\frac{1}{2}(1+\epsilon_{38})^\frac{1}{4}...(1+\epsilon_{2})^{\frac{1}{2^{38}}}(1+\epsilon_{1})^{\frac{1}{2^{39}}}\sqrt[2^{40}]{x}
\]
\[
= \sqrt[2^{40}]{x}\prod_{i=1}^{40}(1+\epsilon_{i})^{\frac{1}{2^{40-i}}}
\]

\item
\[
\frac{\delta \overline{y}}{\delta\epsilon_i}(0,...,\epsilon_i,...,0)
= \frac{\delta}{\delta\epsilon_i}\sqrt[2^{40}]{x}(1+\epsilon_{i})^{\frac{1}{2^{40-i}}}
= \sqrt[2^{40}]{x}\frac{(1+\epsilon_{i})^{\frac{1}{2^{40-i}}-1}}{2^{40-i}}
\]
\[
\frac{\delta \overline{y}}{\delta\epsilon_i}(0,...,0) = \frac{\sqrt[2^{40}]{x}}{2^{40-i}}
\]

\item 
\[
\overline{y} \approx y +
\sum_{i=1}^{40}
\frac{\sqrt[2^{40}]{x}}{2^{40-i}}
\epsilon_i
\]
\[
\overline{y} \approx y +
\sum_{i=1}^{40}
\frac{y}{2^{40-i}}
\epsilon_i
\]

\item
\[
\overline{y}-y \approx
\sum_{i=1}^{40}
\frac{y}{2^{40-i}}
\epsilon_i
\]
\[
\frac{\overline{y}-y}{y} \approx
\sum_{i=1}^{40}
\frac{1}{2^{40-i}}
\epsilon_i
\le \left(\frac{2^{40}-1}{2^{40}}+1\right)\epsilon_{mach}
\]

\end{enumerate}
\item Kwadraten\\
\[
y = x^{2^{40}}
\]
\begin{enumerate}
\item
\[
\overline{y} = fl\left(fl\left(fl\left(fl\left( fl\left(x^2\right)^2 \right)^2 \right)^2 \right)^2 \right)
\]
\[
\overline{y} = (1+\epsilon_{40})\left(...(1+\epsilon_3)\left((1+\epsilon_2)\left((1+\epsilon_1)x^2\right)^2\right)^2...\right)^2
\]
\[
\overline{y} = x^{2^{40}} \sum_{i=1}^{40}(1+\epsilon_{i})^{2^{40-i}}
\]

\item
\[
\frac{\delta \overline{y}}{\delta\epsilon_i}(0,...,\epsilon_i,...,0)
= \frac{\delta}{\delta\epsilon_i}
x^{2^{40}} (1+\epsilon_{i})^{2^{40-i}}
= x^{2^{40}} 2^{40-i}(1+\epsilon_{i})^{2^{40-i}-1}
\]
\[
\frac{\delta \overline{y}}{\delta\epsilon_i}(0,...,0)
= x^{2^{40}} 2^{40-i}
\]

\item
\[
\overline{y} \approx y +
\sum_{i=1}^{40} x^{2^{40}} 2^{40-i} \epsilon_i
= x^{2^{40}} \sum_{i=1}^{40} 2^{40-i} \epsilon_i
\]
\[
\overline{y} \approx y +
y \sum_{i=1}^{40} 2^{40-i} \epsilon_i
\]

\item
\[
\overline{y}-y \approx
y \sum_{i=1}^{40} 2^{40-i} \epsilon_i
\]
\[
\frac{\overline{y}-y}{y} \approx
\sum_{i=1}^{40} 2^{40-i} \epsilon_i
\le \left(2^{40}-1\right)\epsilon_{mach}
\]

\end{enumerate}
\end{enumerate}
We kunnen nu de fout berekenen.
\[
y = x \text{ maar } \overline{y} \neq x
\]
\[
\overline{y} = 
x
\left(
1+\sum_{i=1}^{40}
\frac{1}{2^{40-i}}
\epsilon_i
\right)
\left(
1+
\sum_{j=1}^{40}
2^{40-j} \epsilon_j
\right)
\]

\item Zie matlab
\item Zie matlab
\item Ja, op het eerste deel komen vrij weinig fouten.


\end{enumerate}

\section{Probleem 10}
\begin{enumerate}[(a)]
\item Zie matlab
\item Zie matlab
\[
e^{x^{2}} = \sum_{k=0}^{\infty}\frac{x^{2^{k}}}{k!} \text{, } e^{-x^{2}} = \sum_{k=0}^{\infty}\frac{-x^{2^{k}}}{k!}
\]
\[
f(x) = \frac{e^{x^{2}}-e^{-x^{2}}}{2x^2} = \frac{\sum_{k=0}^{\infty}\frac{x^{2^{k}}}{k!} - \sum_{k=0}^{\infty}\frac{-x^{2^{k}}}{k!}}{2x^2}
= \frac{\sum_{k=0}^{\infty}\frac{2x^{2+4k}}{\frac{2+4k}{2}!}}{2x^2}
\]
\item LOLNOPE

\end{enumerate}

\section{Probleem 11}
\todo{probleem 11}

\end{document}