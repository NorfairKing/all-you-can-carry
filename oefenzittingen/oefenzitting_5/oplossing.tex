\documentclass[12pt,a4paper]{article}
\usepackage[latin1]{inputenc}
\usepackage[dutch]{babel}
\usepackage{amsmath}
\usepackage{amsfonts}
\usepackage{amssymb}
\usepackage{amsthm}
\usepackage{enumerate}
\usepackage{todonotes}

\author{Tom Sydney Kerckhove}
\title{Oefeningen Numerieke Wiskunde:\\ Oefenzitting 5}
\date{11 maart 2014}

\begin{document}
\maketitle

\section{Probleem 1}
Zie de matlab scripts.
Voor de eerste formule bekomen we maximaal $8$ juiste beduidende cijfers, voor de tweede bekomen we er maximaal $11$. We plotten de relatieve fout ten opzichte van $h$ in een loglog-plot en zien duidelijk twee componenten in de fout. Eenderzijds links de afrondingsfouten en anderzijds rechts de benaderingsfouten.

\section{Probleem 2}
De ordes van de benaderingen zijn respectievelijk $1$ en $2$.
\todo{bewijs}

\section{Probleem 3}
\todo{bewijs}

\section{Probleem 4}
\todo{ik zie geen verschil}
\todo{probleem 5 en 6}

\listoftodos

\end{document}