\documentclass[12pt,a4paper]{article}
\usepackage[latin1]{inputenc}
\usepackage[dutch]{babel}
\usepackage{amsmath}
\usepackage{amsfonts}
\usepackage{amssymb}
\usepackage{amsthm}
\usepackage{enumerate}

\author{Tom Sydney Kerckhove}
\title{Oefeningen Numerieke Wiskunde:\\ Oefenzitting 6}
\date{19 maart 2014}

\begin{document}
\maketitle

\section{Probleem 1}
\subsection*{(a)}
\[
f(-1) = -7 \text{ en } f(1) = 1
\]
\subsubsection*{Lagrange}
We zoeken een veelterm van de volgende vorm.
\[
y_1(x) = a_0 + a_1x
\]
Hiervoor lossen we volgend stelsel op.


\subsubsection*{Newton}


\subsection*{(b)}
Het vandermonde stelsel:
\[
\left\{
\begin{array}{r l}
a_0 - 1a_1 &= -7\\
a_0 +  a_2 &= 1\\
\end{array}
\right.
\]
\subsection*{(c)}

\section{Probleem 2}
\section{Probleem 3}
\section{Probleem 4}
\section*{1.}
\[
\sum_{i=1}^{n}\prod_{j=0,j\neq i}^{n}\frac{x-x_j}{x_i-x_j}=1
\]
\begin{proof}

\end{proof}

\section*{2.}
\[
\sum_{i=1}^{n}\prod_{j=0,j\neq i}^{n}\frac{x-x_j}{x_i-x_j}x_{i}^{k}=x^{k}
\]
\begin{proof}

\end{proof}


\end{document}