\documentclass[12pt,a4paper]{article}
\usepackage[latin1]{inputenc}
\usepackage[dutch]{babel}
\usepackage{amsmath}
\usepackage{amsfonts}
\usepackage{amssymb}
\usepackage{amsthm}
\usepackage{enumerate}
\usepackage{tikz}
\usepackage{pgfplots}
\usepackage{todonotes}
\usepackage[left=2cm,right=2cm,top=2cm,bottom=2cm]{geometry}

\author{Tom Sydney Kerckhove}
\title{Oefeningen Numerieke Wiskunde:\\ Oefenzitting 6}
\date{31 maart 2014}

\begin{document}
\maketitle

\section{Probleem 1}
\subsection*{(a)}
\[
f(-1) = -7 \text{ en } f(1) = 1
\]
\subsubsection*{Lagrange}
Bereken eerste de lagrange basis.
\[
l_0(x) = \prod_{j=0,j\neq 0}^{1}\frac{x-x_j}{x_0-x_j} = \frac{x-1}{-1-1} = -\frac{1}{2}x + \frac{1}{2}
\]
\[
l_1(x) = \prod_{j=0,j\neq 1}^{1}\frac{x-x_j}{x_1-x_j} = \frac{x+1}{1+1} = \frac{1}{2}x + \frac{1}{2}
\]
De interpolerende veelterm is dan de volgende:
\[
y_{1}(x) = \sum_{i=0}^{1}l_{i}(x)f(x_i)
= \left(-\frac{1}{2}x + \frac{1}{2}\right)(-7)
+ \left(\frac{1}{2}x + \frac{1}{2}\right)(1)
= 4x-3
\]

\subsubsection*{Newton}
Bereken eerst de newton basis.
\[
b_{0} = f[x_0] = f(x_{0}) = -7
\]
\[
b_{1} = f[x_0,x_1] = \frac{f(x_1)-f(x_0)}{x_1-x_0} = 4
\]
De interpolerende veelterm is dan de volgende:
\[
y_{1}(x) = \sum_{i=0}^1b_i\prod_{j=0}^{i}(x-x_j) = b_0 + b_1(x-x_0) = -7 + 4(x+1) = 4x-3
\]


\subsection*{(b)}
Het vandermonde stelsel:
\[
\left\{
\begin{array}{r l}
a_0 - 1a_1 &= -7\\
a_0 +  a_2 &= 1\\
\end{array}
\right.
\]
Test:
\[
\left\{
\begin{array}{r l}
-3 - 4 &= -7\\
-3 + 4 &= 1\\
\end{array}
\right.
\rightarrow OK
\]
\subsection*{(c)}
\[
E_1(x) = \frac{f^{(2)}(\zeta)}{2!}(x-x_0)(x-x_1)
\]
We weten dat $f_{(2)}(x) = 4$ constant is.
\[
E_1(x) = \frac{4}{2}(x+1)(x-1) = 2x^2-2
\]

\begin{center}
\begin{tikzpicture}
\begin{axis}[
        axis x line=middle, 
        axis y line=middle, 
        ymax=6, ylabel=$E_1(x)$, 
        xlabel=$x$
        ]
    \addplot[domain=-5:5, ] {2*x^2 - 2};
\end{axis}
\end{tikzpicture}
\end{center}
Buiten $[-1,1]$ gedraagt $E_1(x)$ zicht als iets van $O(x^2)$.

\section{Probleem 2}
De tabel del gedeelde differentiaties:
\[
\begin{array}{c|c c c}
-1 & -7 & &\\
 0 & -5 & \frac{-5+7}{0+1} = 2 &\\
 1 &  1 & \frac{1+5}{1+0}  = 6 & \frac{6-2}{1+1} = 2\\
\end{array}
\]
\[
y_2(x) = \sum_{i=0}^{2}f[x_0,...,x_i]\prod_{j=0}^{i-1}(x-x_j)
= -7 + 2(x+1) + 2(x+1)(x-0)
= 2x^2+4x-5
\]
Alternatieve tabel:
\[
\begin{array}{c|c c c}
-1 & -7 & &\\
 0 & -5 & \frac{-5+7}{0+1} = 2 &\\
 1 &  1 & \frac{1+7}{1+1}  = \frac{8}{2} & = 2\\
\end{array}
\]
$E_2(x) = 0$ omdat $y$ en $y_2$ beide van graad twee zijn en $y_2$ uniek is.

\section{Probleem 3}
\[
\begin{array}{c|c c c c}
-1 & -7 & &\\
 0 & -5 & 2 &\\
0.5&-2.5& 5 & 2\\
 1 &  1 & 7 & 2 & 0\\
\end{array}
\]
\[
y_3(x) = \sum_{i=0}^{3}f[x_0,...,x_i]\prod_{j=0}^{i-1}(x-x_j)
\]
\[
= -7 + 2(x+1) + 2(x+1)(x-0) + 0(x+1)(x-0)(x-0.5)
= 2x^2+4x-5
\]
Alternatieve tabel:
\[
\begin{array}{c|c c c c}
-1 & -7 & &\\
 0 & -5 & 2 &\\
0.5&-2.5& 3 & 2\\
 1 &  1 & 4 & 2 & 0\\
\end{array}
\]
We zien dat deze veelterm van graad twee is.
We kunnen immers geen interpolerende veelterm opstellen van graad drie voor een veelterm van graad twee.

\section{Probleem 4}
\section*{1.}
\[
\sum_{i=1}^{n}\prod_{j=0,j\neq i}^{n}\frac{x-x_j}{x_i-x_j}=1
\]
Niet zomaar beginnen uitrekenen, dat wordt een hel.
\begin{proof}
Stel de interpolerende veelterm (van graad $n$) op van de constante functie $y = 1$.
\[
y_n = \sum_{i=0}^{n}l_{i}(x)f(x_i)
\]
We weten dat $f(x_i) = 1$ geldt voor elke $i$.
\[
y_n = \sum_{i=0}^{n}l_{i}(x) \text{ en } y_n(x) = 1
\]
We weten dat $y_n$ $y$ exact benadert omdat $y$ van graad $0$ is en $n\ge 0$. 
\[
y_n = \sum_{i=0}^{n}l_{i}(x) = 1
\]
\end{proof}

\section*{2.}
\[
\sum_{i=1}^{n}l_i(x)x_{i}^{k}=x^{k} \text{ met } k \le n
\]
\begin{proof}
Als we de interpolerende veelterm opstellen van $y=x^k$ wordt deze exact benaderd omdat $k\le n$.
\end{proof}

\section{Probleem 5 tot 8}
\todo{extra}

\section{Probleem 9}
\begin{proof} Bewijs in delen
\begin{itemize}
\item (1) $\rightarrow$ (2) Triviaal: als het voor elke functie geldt geldt het ook voor een veelterm.
\item (2) $\rightarrow$ (3)
We weten:
\[
\int_{a}^{b}p(x)dx = \sum_{i=0}^{n}H_if(x_i)
\]
\[
p(x) = \sum_{i=0}^{n}l_i(x)f(x_i) =
\]
\[
\int_{a}^{b}p(x)dx = \int_{a}^{b}\sum_{i=0}^{n}l_i(x)f(x_i)
\]
\[
= \int_{a}^{b}\sum_{i=0}^{n}l_i(x)f(x_i)dx
= \sum_{i=0}^{n}f(x_i)\int_{a}^{b}l_i(x)dx
\]
\[
\Rightarrow H_i = \int_{a}^{b}l_i(x)dx
\]
\item (3) $\rightarrow$ (1)
\[
\sum_{i=0}^{n}H_if(x_i)
= \sum_{i=0}^{n}f(x_i)\int_{a}^{b}l_i(x)dx
= \int_{a}^{b}\sum_{i=0}^{n}l_i(x)f(x_i)dx
= \int_{a}^{b}y_ndx
\]
\end{itemize}
\end{proof}

\section{Probleem 10}
We weten dat de gewichten $H_i$ de integralen zijn van de lagrange veeltermen $l_i$.
\[
H_i = \int_{a}^{b}l_idx
\]
We kunnen de gewichten dus gewoon berekenen.
\[
H_0 = \int_{-1}^{1}l_{0}dx = \int_{-1}^{1}\frac{x - \frac{1}{2}}{-\frac{1}{2}-\frac{1}{2}} = \int_{-1}^{1}\left(\frac{1}{2}-x\right)dx = 1
\]
\[
H_1 = \int_{-1}^{1}l_{1}dx = \int_{-1}^{1}\frac{x + \frac{1}{2}}{\frac{1}{2}+\frac{1}{2}} = \int_{-1}^{1}\left(\frac{1}{2}+x\right)dx = 1
\]

We kunnen nu nakijken tot welke graad de integraal exact wordt berekend.
\[
(d=0):\ 1+1=2=\int_{-1}^{1}dx \rightarrow OK
\]
\[
(d=1):\ -\frac{1}{2}+\frac{1}{2}=0=\int_{-1}^{1}xdx \rightarrow OK
\]
\[
(d=2):\ \frac{1}{4}+\frac{1}{4}=\frac{1}{2}\neq\int_{-1}^{1}x^2dx \rightarrow NOK
\]
De nauwkeurigheidsgraad is dus $1$.
Nauwkeurigheidsgraad $ = 1 \ge n$ dus interpolerend.

\section{Probleem 11}

\subsection*{Alternatieve methode}
\todo{waarom werkt dit?}
\[
\left\{
\begin{array}{cccc}
H_0 &+ H_1 &= \int_{-1}^{1}dx &= 2\\
-\frac{1}{2}H_0 &+ \frac{1}{2}H_1 &= \int_{-1}^{1}xdx &= 0\\
\end{array}
\right.
\]
We vinden voor $H_0$ en $H_1$ beide $1$.




\end{document}