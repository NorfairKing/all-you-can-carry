\documentclass[12pt,a4paper]{article}
\usepackage[latin1]{inputenc}
\usepackage[dutch]{babel}
\usepackage{amsmath}
\usepackage{amsfonts}
\usepackage{amssymb}
\usepackage{amsthm}
\usepackage{enumerate}
\setlength{\parindent}{0pt}

\author{Tom Sydney Kerckhove}
\title{Oefeningen Numerieke Wiskunde:\\ Oefenzitting 12}
\date{23 april 2014}

\begin{document}
\maketitle

\section{Probleem 1}
Te Bewijzen:
\[
\lim_{k\rightarrow\infty}\frac{\epsilon^{(k+1)}}{(\epsilon^{(k)})^{n}} = \rho_{p}
\]
\begin{proof}
\[
F(x) = x - \frac{-\cos(x)}{-\sin(x)} = x + \cot x
\]
Zie p 323 (bewijs)
\[
F'(x) = 1  - \csc^{2}(x) \rightarrow F'(x^{*}) = 0
\]
\[
F''(x) = 2\cot(x)\csc^2(x) \rightarrow F''(x^{*})= 0
\]
\[
F'''(x) = -2 (\cos(2x)+2) \csc^4(x) \rightarrow F'''(x^{*}) = -2
\]
Het proces is dus van orde $3$.
\end{proof}
KOMT OP EXAMEN

\end{document}