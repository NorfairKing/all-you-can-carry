\documentclass[12pt,a4paper]{article}
\usepackage[latin1]{inputenc}
\usepackage[dutch]{babel}
\usepackage{amsmath}
\usepackage{amsfonts}
\usepackage{amssymb}
\usepackage{amsthm}
\usepackage{enumerate}
\usepackage{tikz}
\usepackage{pgfplots}
\usepackage{todonotes}
\usepackage[left=2cm,right=2cm,top=2cm,bottom=2cm]{geometry}

\author{Tom Sydney Kerckhove}
\title{Oefeningen Numerieke Wiskunde:\\ Oefenzitting 11}
\date{7 april 2014}

\begin{document}
\maketitle

\todo{probleem 1}
\section{Probleem 2}
\subsection*{(a)}
\begin{verbatim}
http://www.wolframalpha.com/input/?i=-+log+x
\end{verbatim}
\[
x + \log(x) = 0 \Leftrightarrow x = -\log(x)
\]
We spreken dus van volledige convergentie.
$F'(x) = \frac{1}{x}$

\subsection*{(b)}
\begin{verbatim}
http://www.wolframalpha.com/input/?i=-e^x
\end{verbatim}
\[
x + \log(x) = 0 \Leftrightarrow e^{-x}
\]

\end{document}