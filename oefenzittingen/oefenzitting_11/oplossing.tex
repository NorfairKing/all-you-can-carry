\documentclass[12pt,a4paper]{article}
\usepackage[latin1]{inputenc}
\usepackage[dutch]{babel}
\usepackage{amsmath}
\usepackage{amsfonts}
\usepackage{amssymb}
\usepackage{amsthm}
\usepackage{enumerate}
\usepackage{tikz}
\usepackage{pgfplots}
\usepackage{todonotes}
\usepackage[left=2cm,right=2cm,top=2cm,bottom=2cm]{geometry}

\author{Tom Sydney Kerckhove}
\title{Oefeningen Numerieke Wiskunde:\\ Oefenzitting 11}
\date{7 april 2014}

\begin{document}
\maketitle

\section{Probleem 1}
Zie verbetering van Toledo.
\section{Probleem 2}
\subsection*{(a)}
\subsubsection*{Consistentie}
\begin{verbatim}
http://www.wolframalpha.com/input/?i=-+log+x
\end{verbatim}
\[
x + \log(x) = 0 \Leftrightarrow x = -\log(x)
\Leftrightarrow F(x) = x
\]
We spreken dus van volledige consistentie.
\subsubsection*{Convergentie}
$$|F'(x^*)| = |\frac{1}{-x^*}| = |\frac{1}{0.56714}| \approx 1.7632 > 1$$
Er gebeurt hier dus spiraalvorigme divergentie.

\subsection*{(b)}
\subsubsection*{Consistentie}
\begin{verbatim}
http://www.wolframalpha.com/input/?i=-e^x
\end{verbatim}
\[
x + \log(x) = 0 \Leftrightarrow x = e^{-x} 
\Leftrightarrow F(x) = x
\]
$$\Leftrightarrow -x = \log(x) \Leftrightarrow f(x) = 0$$
Er is op nieuw volledige consistentie.
\subsubsection*{Convergentie}
$$|F'(x^*)| = |-e^{-x^*}| = e^{-0.56714} < 1$$
Hier hebben we dus spiraalvormige convergentie want de afgeleide is kleiner dan nul.
\end{document}