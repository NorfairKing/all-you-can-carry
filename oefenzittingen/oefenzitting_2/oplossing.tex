\documentclass[12pt,a4paper]{article}
\usepackage[latin1]{inputenc}
\usepackage[dutch]{babel}
\usepackage{amsmath}
\usepackage{amsfonts}
\usepackage{amssymb}
\usepackage{amsthm}
\author{Tom Sydney Kerckhove}
\title{Oefeningen Numerieke Wiskunde:\\ Oefenzitting 2}
\begin{document}
\maketitle

\section{Probleem 1}
De exponent gaat van $-126$ tot $127$. Dit betekent dat hier $8$ bits voor nodig zijn. Er blijven dus nog $32-8 = 23$ bits over voor de mantisse want er is ook nog een tekenbit.
Een andere manier om dit resultaat te bekomen is de formule voor de machine precisie te gebruiken.
\[
\epsilon_{mach} = \frac{b^{1-p}}{2}
\]
Hierin is $b$ de basis en $p$ het aantal juiste beduidende cijfers.
Uit deze formule halen we de $p$.
\[
p 1-\log_{b}(2\epsilon_{mach})
\]
\[
p = 1-\log_{2}(2\cdot 2^{-24}) = 23
\]
Volgens dezelfde redenering vinden we dat het aantal bits voor het teken, de mantisse en de exponent bij de dubbele-nauwkeurigheidsgetallen respectievelijk $1$, $11$ en $52$ zijn.

\section{Probleem 2}
\begin{itemize}
\item We weten dat $1$ en $2$ respectievelijk als volgt voorgesteld worden.
\[1 = .100\ldots00 \cdot 2^1\]
\[2 = .100\ldots00 \cdot 2^2\]
Hier zitten dus $2^m-1=2^{52}-1$ getallen tussen.
\item We weten dat $7$ en $9$ respectievelijk als volgt voorgesteld worden.
\[7 = .11100\ldots00 \cdot 2^3\]
\[9 = .10010\ldots00 \cdot 2^4\]
Hier zitten dus $2^{50}+2^{49}-1$ getallen tussen.
\end{itemize}

\section{Probleem 3}
Het laatste zinvolle getal in deze rij is $1000$. De overgang van $999$ naar $1000$ resulteert niet in een fout want $1000$ wordt als volgt voorgesteld.
\[0.100 \cdot 10^4\]
Als we hier echter nog $1$ bij optellen gebeurt er een absolute afrondingsfout van precies $-1$. Vanaf dan is de beschreven rij dus constant ($1000$).

\section{Probleem 4}
\begin{itemize}
\item
\begin{itemize}
\item \textbf{bepaal\_b}
We beginnen met $A=1$, dan vermenigvuldigen we $A$ met $2$ tot er een afrondingsfout gebeurd wanneer je $(A+1)-A=1$ evalueert. 
$A$ is nu gelijk aan $2^{10}$. Vervolgens initialiseren we $i$ op $1$ en hogen $i$ op tot $(A+i)=A$ geen afrondingsfout meer geeft. Dit is wanneer $i$ $6$ wordt. $b$ is dus $.103\cdot 10^4-.102\cdot 10^4=10$.
\item \textbf{bepaal\_p}
We initialiseren $p$ op $1$ en $z$ op $10$. Nu hogen we $p$ en de exponent van $10^1$ op tot $(z+1)-z=1$ een afrondingsfout geeft.
Dit is wanneer $p=3$ geldt. 
\end{itemize}
\item
\begin{itemize}
\item \textbf{bepaal\_b}\\
\textbf{Te Bewijzen}
\begin{proof}
TODO
\end{proof}
\item \textbf{bepaal\_p}\\
\textbf{Te Bewijzen}
\begin{proof}
TODO
\end{proof}
\end{itemize}
\end{itemize}

\section{Probleem 5}
\[
\overline{y} =
fl\left(
\frac{x}{fl\left(fl\left(\sqrt{fl\left(x+1\right)}\right)+1\right)}
\right)
\]
\[
=
\frac{x(1+\epsilon_1)}{\left(\left(\sqrt{\left(x+1\right)(1+\epsilon_4)}\right)(1+\epsilon_3)+1\right)(1+\epsilon_2)}
\]
\[
\overline{y} = F(0,0,0,0) + \sum_{i=1}^4\epsilon_i\frac{\delta F}{\delta\epsilon_i}(0,0,0,0)
\]
\[
\frac{\delta F}{\delta \epsilon_1}(\epsilon_1,0,0,0)
= \frac{x\epsilon_1}{\sqrt{x+1}+1}
\]
\[
\frac{\delta F}{\delta \epsilon_2}(0,\epsilon_2,0,0)
= \frac{-x}{\left(\sqrt{x+1}+1\right)(1+\epsilon_2)^2}
\]
\[
\frac{\delta F}{\delta \epsilon_3}(0,0,\epsilon_3,0)
...
\]
TODO\\
Oplossing:
\[
\frac{\overline{y}-y}{y} \approx -\frac{1}{2}\frac{\sqrt{1+x}}{\sqrt{1+x}+1}\epsilon_1 - \frac{\sqrt{1+x}}{\sqrt{1+x}+1}\epsilon_2 - \epsilon_3+\epsilon_4
\]

\section{Probleem 6}
\[
\overline{y}
=
fl
\left(
\frac{
fl\left(
1-fl(\cos(x))
\right)}
{fl(x^2)}
\right)
\]
\[
=
\left(
\frac{
\left(
1-(\cos(x)(1+\epsilon_3))
\right)(1+\epsilon_2)}
{(x^2)(1+\epsilon_4)}
\right)(1+\epsilon_1)
\]
TODO

\section{Probleem 7}
\[
\overline{S} = fl(fl(...fl(fl(fl(fl(a_1+a_2)+a_3)+a_4)...)+a_n)
\]
\[
= ((...((((a_1+a_2)(1+\epsilon_2))+a_3)(1+\epsilon_3))...)+a_n)(1+\epsilon_n)
\]
\[
\overline{S}-S = \sum_{i=1}^na_i - ((...((((a_1+a_2)(1+\epsilon_2))+a_3)(1+\epsilon_3))...)+a_n)(1+\epsilon_n)
\]
\[
\approx 
\]
TODO hoe


\end{document}