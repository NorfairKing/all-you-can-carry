\documentclass[12pt,a4paper]{article}
\usepackage[latin1]{inputenc}
\usepackage[dutch]{babel}
\usepackage{amsmath}
\usepackage{amsfonts}
\usepackage{amssymb}
\usepackage{amsthm}
\usepackage{enumerate}
\author{Tom Sydney Kerckhove}
\title{Oefeningen Numerieke Wiskunde:\\ Oefenzitting 2}
\date{19 februari 2014}

\begin{document}
\maketitle

\section{Probleem 1}
De exponent gaat van $-126$ tot $127$. Dit betekent dat hier $8$ bits voor nodig zijn. Er blijven dus nog $32-8 = 23$ bits over voor de mantisse want er is ook nog een tekenbit.
Een andere manier om dit resultaat te bekomen is de formule voor de machine precisie te gebruiken.
\[
\epsilon_{mach} = \frac{b^{1-p}}{2}
\]
Hierin is $b$ de basis en $p$ het aantal juiste beduidende cijfers.
Uit deze formule halen we de $p$.
\[
p 1-\log_{b}(2\epsilon_{mach})
\]
\[
p = 1-\log_{2}(2\cdot 2^{-24}) = 23
\]
Volgens dezelfde redenering vinden we dat het aantal bits voor het teken, de mantisse en de exponent bij de dubbele-nauwkeurigheidsgetallen respectievelijk $1$, $11$ en $52$ zijn.

\section{Probleem 2}
\begin{itemize}
\item We weten dat $1$ en $2$ respectievelijk als volgt voorgesteld worden.
\[1 = .100\ldots00 \cdot 2^1\]
\[2 = .100\ldots00 \cdot 2^2\]
Hier zitten dus $2^m-1=2^{52}-1$ getallen tussen.
\item We weten dat $7$ en $9$ respectievelijk als volgt voorgesteld worden.
\[7 = .11100\ldots00 \cdot 2^3\]
\[9 = .10010\ldots00 \cdot 2^4\]
Hier zitten dus $2^{50}+2^{49}-1$ getallen tussen.
\end{itemize}

\section{Probleem 3}
Het laatste zinvolle getal in deze rij is $1000$. De overgang van $999$ naar $1000$ resulteert niet in een fout want $1000$ wordt als volgt voorgesteld.
\[0.100 \cdot 10^4\]
Als we hier echter nog $1$ bij optellen gebeurt er een absolute afrondingsfout van precies $-1$. Vanaf dan is de beschreven rij dus constant ($1000$).

\section{Probleem 4}
\begin{itemize}
\item
\begin{itemize}
\item \textbf{bepaal\_b}
We beginnen met $A=1$, dan vermenigvuldigen we $A$ met $2$ tot er een afrondingsfout gebeurd wanneer je $(A+1)-A=1$ evalueert. 
$A$ is nu gelijk aan $2^{10}$. Vervolgens initialiseren we $i$ op $1$ en hogen $i$ op tot $(A+i)=A$ geen afrondingsfout meer geeft. Dit is wanneer $i$ $6$ wordt. $b$ is dus $.103\cdot 10^4-.102\cdot 10^4=10$.
\item \textbf{bepaal\_p}
We initialiseren $p$ op $1$ en $z$ op $10$. Nu hogen we $p$ en de exponent van $10^1$ op tot $(z+1)-z=1$ een afrondingsfout geeft.
Dit is wanneer $p=3$ geldt. 
\end{itemize}
\item
\begin{itemize}
\item \textbf{bepaal\_b}\\
\textbf{Te Bewijzen}
\begin{proof}
TODO
\end{proof}
\item \textbf{bepaal\_p}\\
\textbf{Te Bewijzen}
\begin{proof}
TODO
\end{proof}
\end{itemize}
\end{itemize}

\section{Probleem 5}
\begin{enumerate}
\item
\[
\overline{y} =
fl\left(
\frac{x}{fl\left(fl\left(\sqrt{fl\left(x+1\right)}\right)+1\right)}
\right)
\]
\[
=
\frac{x(1+\epsilon_1)}{\left(\left(\sqrt{\left(x+1\right)(1+\epsilon_4)}\right)(1+\epsilon_3)+1\right)(1+\epsilon_2)}
\]
\[
\overline{y} = y + \sum_{i=1}^4\epsilon_i\frac{\delta F}{\delta\epsilon_i}(0,0,0,0)
\]
\item
\begin{itemize}
\item
\[
\frac{\delta \overline{y}}{\delta \epsilon_1}(\epsilon_1,0,0,0)
= \frac{x}{\sqrt{x+1}+1}
\]
\item
\[
\frac{\delta \overline{y}}{\delta \epsilon_2}(0,\epsilon_2,0,0)
= \frac{-x}{\left(\sqrt{x+1}+1\right)(1+\epsilon_2)^2}
\]
\[
\frac{\delta \overline{y}}{\delta \epsilon_2}(0,0,0,0)
= \frac{-x}{\sqrt{x+1}+1}
\]
\item
\[
\frac{\delta \overline{y}}{\delta \epsilon_3}(0,0,\epsilon_3,0)
= \frac{-x\sqrt{1+x}}{\left((\sqrt{1+x})(1+\epsilon_3)+1\right)^2}
\]
\[
\frac{\delta \overline{y}}{\delta \epsilon_3}(0,0,0,0)
= \frac{-x\sqrt{1+x}}{\left(\sqrt{1+x}+1\right)^2}
\]
\item
\[
\frac{\delta \overline{y}}{\delta \epsilon_4}(0,0,0,\epsilon_4)
= \frac{-x(x+1)}{2\sqrt{(x+1)(1+\epsilon_4)}\left(\sqrt{(x+1)(1+\epsilon_4)}+1\right)^2}
\]
\[
\frac{\delta \overline{y}}{\delta \epsilon_4}(0,0,0,0)
= \frac{-x(x+1)}{2\sqrt{x+1}\left(\sqrt{x+1}+1\right)^2}
\]
\end{itemize}
\item
\[
\overline{y} =
\frac{x}{\sqrt{x+1}+1} \epsilon_1
+
\frac{-x}{\sqrt{x+1}+1} \epsilon_2
+
\frac{-x\sqrt{1+x}}{\left(\sqrt{1+x}+1\right)^2} \epsilon_3
+
\frac{-x(x+1)}{2\sqrt{x+1}\left(\sqrt{x+1}+1\right)^2} \epsilon_4
\]
\[
= 
y \epsilon_1
-
y \epsilon_2
+
y\frac{-\sqrt{1+x}}{\sqrt{1+x}+1} \epsilon_3
+
y
\frac{-(x+1)}{2\sqrt{x+1}\left(\sqrt{x+1}+1\right)} \epsilon_4
\]
\[
\overline{y}-y
= 
\epsilon_1
-
\epsilon_2
+
\frac{-\sqrt{1+x}}{\sqrt{1+x}+1} \epsilon_3
+
\frac{-(x+1)}{2\sqrt{x+1}\left(\sqrt{x+1}+1\right)} \epsilon_4
\]
\[
\frac{\overline{y} - y}{y}
=
\frac{\sqrt{x+1}+1}{x}
\epsilon_1
-
\frac{\sqrt{x+1}+1}{x}
\epsilon_2
+
\frac{-\sqrt{1+x}}{x} \epsilon_3
+
\frac{-(x+1)}{2x\sqrt{x+1}} \epsilon_4
\]
\end{enumerate}

TODO\\
Oplossing:
\[
\frac{\overline{y}-y}{y} \approx -\frac{1}{2}\frac{\sqrt{1+x}}{\sqrt{1+x}+1}\epsilon_1 - \frac{\sqrt{1+x}}{\sqrt{1+x}+1}\epsilon_2 - \epsilon_3+\epsilon_4
\]

\section{Probleem 6}

\[
y = \frac{1-\cos(x)}{x^2}
\]
\begin{enumerate}
\item
\[
\overline{y} = fl\left(\frac{fl(1-fl(\cos(x)))}{fl(x^2)}\right)
\]
\[
\overline{y} = \frac{(1-(\cos(x))(1+\epsilon_1))(1+\epsilon_2)}{(x^2)(1+\epsilon_3)}(1+\epsilon_4)
\]

\item
\[
\frac{\delta \overline{y}}{\delta \epsilon_1}(\epsilon_1,0,0,0)
= \frac{\delta}{\delta \epsilon_1}\frac{1-(\cos(x))(1+\epsilon_1)}{x^2} 
= -\frac{\cos(x)}{x^2}
\]
\[
\frac{\delta \overline{y}}{\delta \epsilon_2}(0,\epsilon_2,0,0)
= \frac{\delta}{\delta \epsilon_2}\frac{(1-\cos(x))(1+\epsilon_2)}{x^2}
= \frac{(1-\cos(x))}{x^2}
\]
\[
\frac{\delta \overline{y}}{\delta \epsilon_3}(0,0,\epsilon_3,0)
= \frac{\delta}{\delta \epsilon_3}\frac{1-\cos(x)}{(x^2)(1+\epsilon_3)} = \frac{1-\cos(x)}{(x^2)}\frac{-1}{(1+\epsilon_3)^2}
\]
\[
\frac{\delta \overline{y}}{\delta \epsilon_3}(0,0,0,0) = -\frac{1-\cos(x)}{x^2}
\]
\[
\frac{\delta \overline{y}}{\delta \epsilon_4}(0,0,0,\epsilon_4)
= \frac{\delta}{\delta \epsilon_4}\frac{(1-\cos(x))(1+\epsilon_4)}{x^2}
= \frac{1 - \cos(x)}{x^2}
\]

\item
\[
\overline{y} \approx y -\frac{\cos(x)}{x^2}\epsilon_1
+ \frac{(1-\cos(x))}{x^2}\epsilon_2
- \frac{1-\cos(x)}{x^2}\epsilon_3
+ \frac{1 - \cos(x)}{x^2}\epsilon_4
\]
\[
\overline{y} \approx y - \frac{\cos(x)}{x^2}\epsilon_1 + y\epsilon_2 - y\epsilon_3 + y\epsilon_4
\]
\item
\[
\overline{y}-y \approx - \frac{\cos(x)}{x^2}\epsilon_1 + y\epsilon_2 - y\epsilon_3 + y\epsilon_4
\]
\[
\frac{\overline{y}-y}{y} \approx - \frac{\cos(x)}{x(1-\cos(x))}\epsilon_1 + \epsilon_2 - \epsilon_3 + \epsilon_4
\]
\end{enumerate}

\section{Probleem 7}
\[
S = \sum_{i=1}^na_i
\]
\begin{enumerate}
\item 
\[
\overline{S} = fl(fl(...fl(fl(fl(fl(a_1+a_2)+a_3)+a_4)...)+a_n)
\]
\[
= ((...((((a_1+a_2)(1+\epsilon_2))+a_3)(1+\epsilon_3))...)+a_n)(1+\epsilon_n)
\]

\begin{itemize}
\item
\[
\frac{\delta\overline{y}}{\delta \epsilon_2}(\epsilon_2,0,...,0)= a_1 + a_2
\]
\item
\[
\frac{\delta\overline{y}}{\delta \epsilon_3}(0,\epsilon_3,0,...,0)= a_1 + a_2 + a_3 
\]
\item
\[
\frac{\delta\overline{y}}{\delta \epsilon_i}(0,...,0,\epsilon_i,0,...,0) = \sum_{j=1}^ia_j
\]
\end{itemize}

\item
\[
\overline{S} \approx y + \sum_{k=2}^n\epsilon_k\sum_{i=1}^ka_i
\]

\item
\[
\overline{S}-S \approx \sum_{i=2}^n\epsilon_i\sum_{j=1}^na_j
\]
\end{enumerate}

\section{Probleem 8}
\begin{enumerate}[(a)]
\item
\[
y=x\sin(x)
\]
\begin{enumerate}[1.]
\item
\[
\overline{y} = fl(xfl(\sin(x)))
\]
\[
\overline{y} = (x\sin(x)(1+\epsilon_1))(1+\epsilon_2)
\]
\item
\begin{itemize}
\item
\[
\frac{\delta\overline{y}}{\delta\epsilon_1}(\epsilon_1,0) = x\sin(x)
\]
\item
\[
\frac{\delta\overline{y}}{\delta\epsilon_2}(0,\epsilon_2) = x\sin(x)
\]
\end{itemize}
\item
\[
\overline{y} \approx y + x\sin(x)\epsilon_1 + x\sin(x)\epsilon_2
\]
\[
\overline{y} \approx y + y\epsilon_2 + y\epsilon_3
\]
\item
\[
\overline{y} - y = y\epsilon_1 + y\epsilon_2
\]
\[
\frac{\overline{y}-y}{y} = \epsilon_1 + \epsilon_2
\]
\end{enumerate}


\item
\[
y= \frac{1-\cos(x)}{\sin(x)}
\]
\begin{enumerate}[1.]
\item
\[
\overline{y} = fl\left(\frac{fl(1-fl(\cos(x)))}{fl(\sin(x))}\right)
\]
\[
\overline{y} = \left(\frac{(1-(\cos(x))(1+\epsilon_1))(1+\epsilon_2)}{(\sin(x))(1+\epsilon_3)}\right)(1+\epsilon_4)
\]
\item
\begin{itemize}
\item
\[
\frac{\delta\overline{y}}{\delta\epsilon_1}(\epsilon_1,0,0,0)
= \frac{\delta\overline{y}}{\delta\epsilon_1}
\frac{1-\cos(x)(1+\epsilon_1)}{\sin(x)}
= \frac{-\cos(x)}{\sin(x)}
\]
\item
\[
\frac{\delta\overline{y}}{\delta\epsilon_2}(0,\epsilon_2,0,0)
= \frac{\delta\overline{y}}{\delta\epsilon_2}
\frac{(1-\cos(x))(1+\epsilon_2)}{\sin(x)}
= \frac{1-\cos(x)}{\sin(x)}
\]
\item
\[
\frac{\delta\overline{y}}{\delta\epsilon_3}(0,0,\epsilon_3,0)
= \frac{\delta\overline{y}}{\delta\epsilon_3} 
\frac{1-\cos(x)}{(\sin(x))(1+\epsilon_3)}
= \frac{\cos(x)-1}{\sin(x)(1+\epsilon_3)^2}
\]
\[
\frac{\delta\overline{y}}{\delta\epsilon_3}(0,0,0,0)
= \frac{\cos(x)-1}{\sin(x)}
\]
\item
\[
\frac{\delta\overline{y}}{\delta\epsilon_4}(0,0,0,\epsilon_4)
= \frac{\delta\overline{y}}{\delta\epsilon_4} 
\left(\frac{1-\cos(x)}{\sin(x)}\right)(1+\epsilon_4)
= \frac{1-\cos(x)}{\sin(x)}
\]
\end{itemize}
\item
\[
\overline{y} \approx y +
\frac{-\cos(x)}{\sin(x)} \epsilon_1
+
\frac{1-\cos(x)}{\sin(x)} \epsilon_2
+
\frac{\cos(x)-1}{\sin(x)} \epsilon_3
+
\frac{1-\cos(x)}{\sin(x)} \epsilon_4
\]
\[
\overline{y} \approx y +
\frac{-\cos(x)}{\sin(x)} \epsilon_1
+ y \epsilon_2
- y \epsilon_3
+ y \epsilon_4
\]
\item
\[
\overline{y} - y \approx 
\frac{-\cos(x)}{\sin(x)} \epsilon_1
+ y \epsilon_2
- y \epsilon_3
+ y \epsilon_4
\]
\[
\frac{\overline{y}-y}{y} \approx 
\frac{-\cos(x)}{1-\cos(x)} \epsilon_1
+ \epsilon_2
- \epsilon_3
+ \epsilon_4
\]
\end{enumerate}

\item
\[
y= \frac{1-e^{-2x}}{x}
\]
\begin{enumerate}[1.]
\item
\[
\overline{y} = fl\left(\frac{fl\left(1-fl(e^{fl(-2x)})\right)}{x}\right)
\]
\[
\overline{y} = \left(\frac{\left(1-((e^{(-2x)(1+\epsilon_1)})(1+\epsilon_2))\right)(1+\epsilon_3)}{x}\right)(1+\epsilon_4)
\]
\item
\begin{itemize}
\item
\[
\frac{\delta\overline{y}}{\delta\epsilon_1}(\epsilon_1,0,0,0)
= \frac{\delta\overline{y}}{\delta\epsilon_1}
\frac{1-e^{(-2x)(1+\epsilon_1)}}{x}
= \frac{2xe^{(-2x)(1+\epsilon_1)}}{x}
\]
\[
\frac{\delta\overline{y}}{\delta\epsilon_1}(0,0,0,0)
= \frac{2xe^{-2x}}{x}
\]
\item
\[
\frac{\delta\overline{y}}{\delta\epsilon_2}(0,\epsilon_2,0,0)
= \frac{\delta\overline{y}}{\delta\epsilon_2}
\frac{1-(e^{-2x})(1+\epsilon_2)}{x}
= \frac{-e^{-2x}}{x}
\]
\item
\[
\frac{\delta\overline{y}}{\delta\epsilon_3}(0,0,\epsilon_3,0)
= \frac{\delta\overline{y}}{\delta\epsilon_3}
\frac{(1-e^{-2x})(1+\epsilon_3)}{x}
= \frac{1-e^{-2x}}{x}
\]

\item
\[
\frac{\delta\overline{y}}{\delta\epsilon_4}(0,0,0,\epsilon_4)
= \frac{\delta\overline{y}}{\delta\epsilon_4}
\left(\frac{1-e^{-2x}}{x}\right)(1+\epsilon_4)
= \frac{1-e^{-2x}}{x}
\]
\end{itemize}
\item
\[
\overline{y} \approx y +
\frac{2xe^{-2x}}{x}
+
\frac{-e^{-2x}}{x}
+
\frac{1-e^{-2x}}{x}
+
\frac{1-e^{-2x}}{x}
\]
\[
\overline{y} \approx y +
\frac{2xe^{-2x}}{x} \epsilon_1
+
\frac{-e^{-2x}}{x} \epsilon_2
+
y \epsilon_3
+
y \epsilon_4
\]
\item
\[
\overline{y} - y \approx 
\frac{2xe^{-2x}}{x} \epsilon_1
+
\frac{-e^{-2x}}{x} \epsilon_2
+
y \epsilon_3
+
y \epsilon_4
\]
\[
\frac{\overline{y}-y}{y} \approx 
\frac{2xe^{-2x}}{1-e^{-2x}}\epsilon_1
+
\frac{-e^{-2x}}{1-e^{-2x}}\epsilon_2
+
\epsilon_3
+
\epsilon_4
\]
\end{enumerate}

\item
\[
y= (1+x)^{\frac{1}{x}}
\]
\begin{enumerate}[1.]
\item
\[
\overline{y} = fl\left((1+x)^{\frac{1}{x}}\right)
\]
\[
\overline{y} = \left(((1+x)(1+\epsilon_2))^{\left(\frac{1}{x}\right)(1+\epsilon_1)}\right)(1+\epsilon_3)
\]
\item
\begin{itemize}
\item
\[
\frac{\delta\overline{y}}{\delta\epsilon_1}(\epsilon_1,0,0,0)
= \frac{\delta\overline{y}}{\delta\epsilon_1}
(1+x)^{\left(\frac{1}{x}\right)(1+\epsilon_1)}
= \frac{(x+1)^{\frac{\epsilon_1+1}{x}}ln(x+1)}{x}
\]
\[
\frac{\delta\overline{y}}{\delta\epsilon_1}(0,0,0,0)
= \frac{(x+1)^{\frac{1}{x}}ln(x+1)}{x}
\]
\item
\[
\frac{\delta\overline{y}}{\delta\epsilon_2}(0,\epsilon_2,0,0)
= \frac{\delta\overline{y}}{\delta\epsilon_2}
((1+x)(1+\epsilon_2))^{\frac{1}{x}}
= \frac{(1+x)}{a}((1+x)(1+\epsilon_2))^{\frac{1}{x}-1}
\]
\[
\frac{\delta\overline{y}}{\delta\epsilon_2}(0,0,0,0)
= \frac{(1+x)}{a}(1+x)^{\frac{1}{x}-1}
=\frac{(1+x)^{\frac{1}{x}}}{a}
\]
\item
\[
\frac{\delta\overline{y}}{\delta\epsilon_3}(0,0,\epsilon_3,0)
= \frac{\delta\overline{y}}{\delta\epsilon_3}
\left((1+x)^{\frac{1}{x}}\right)(1+\epsilon_3)
= \left((1+x)^{\frac{1}{x}}\right)
\]

\end{itemize}
\item
\[
\overline{y} \approx y +
\frac{(x+1)^{\frac{1}{x}}ln(x+1)}{x} \epsilon_1
+
\frac{(1+x)^{\frac{1}{x}}}{a}\epsilon_2
+
\left((1+x)^{\frac{1}{x}}\right) \epsilon_3
\]
\[
\overline{y} \approx y +
\frac{ln(x+1)}{x}y \epsilon_1
+ \frac{1}{a}y \epsilon_2
+ y \epsilon_3
\]
\item
\[
\overline{y} - y \approx 
\frac{ln(x+1)}{x}y \epsilon_1
+ \frac{1}{a}y \epsilon_2
+ y \epsilon_3
\]
\[
\frac{\overline{y}-y}{y} \approx 
\frac{ln(x+1)}{x} \epsilon_1
+ \frac{1}{a} \epsilon_2
+ \epsilon_3
\]
\end{enumerate}

TODO
\item
\[
y= 
\]
\begin{enumerate}[1.]
\item
\[
\overline{y} = ...
\]
\[
\overline{y} = ...
\]
\item
\begin{itemize}
\item
\[
\frac{\delta\overline{y}}{\delta\epsilon_1}(\epsilon_1,0,0,0)
= \frac{\delta\overline{y}}{\delta\epsilon_1}
...
= 
\]
\[
\frac{\delta\overline{y}}{\delta\epsilon_1}(0,0,0,0)
= ...
\]
\item
\[
\frac{\delta\overline{y}}{\delta\epsilon_2}(0,\epsilon_2,0,0)
= \frac{\delta\overline{y}}{\delta\epsilon_2}
...
= 
\]
\item
\[
\frac{\delta\overline{y}}{\delta\epsilon_3}(0,0,\epsilon_3,0)
= \frac{\delta\overline{y}}{\delta\epsilon_3}
...
= 
\]

\item
\[
\frac{\delta\overline{y}}{\delta\epsilon_4}(0,0,0,\epsilon_4)
= \frac{\delta\overline{y}}{\delta\epsilon_4}
...
= 
\]
\end{itemize}
\item
\[
\overline{y} \approx y +
...
\]
\[
\overline{y} \approx y +
...
\]
\item
\[
\overline{y} - y \approx 
...
\]
\[
\frac{\overline{y}-y}{y} \approx 
...
\]
\end{enumerate}

\end{enumerate}

\section{Probleem 9}
\section{Probleem 10}
\section{Probleem 11}

\end{document}