\documentclass[12pt,a4paper]{article}
\usepackage[latin1]{inputenc}
\usepackage[dutch]{babel}
\usepackage{amsmath}
\usepackage{amsfonts}
\usepackage{amssymb}
\author{Tom Sydney Kerckhove}
\title{Oefeningen Numerieke Wiskunde:\\ Oefenzitting 2}
\begin{document}
\maketitle

\section{Probleem 1}
De exponent gaat van $-126$ tot $127$. Dit betekent dat hier $8$ bits voor nodig zijn. Er blijven dus nog $32-8 = 23$ bits over voor de mantisse want er is ook nog een tekenbit.
Een andere manier om dit resultaat te bekomen is de formule voor de machine precisie te gebruiken.
\[
\epsilon_{mach} = \frac{b^{1-p}}{2}
\]
Hierin is $b$ de basis en $p$ het aantal juiste beduidende cijfers.
Uit deze formule halen we de $p$.
\[
p 1-\log_{b}(2\epsilon_{mach})
\]
\[
p = 1-\log_{2}(2\cdot 2^{-24}) = 23
\]
Volgens dezelfde redenering vinden we dat het aantal bits voor het teken, de mantisse en de exponent bij de dubbele-nauwkeurigheidsgetallen respectievelijk $1$, $11$ en $52$ zijn.

\section{Probleem 2}
\begin{itemize}
\item We weten dat $1$ en $2$ respectievelijk als volgt voorgesteld worden.
\[1 = .100\ldots00 \cdot 2^1\]
\[2 = .100\ldots00 \cdot 2^2\]
Hier zitten dus $2^m-1=2^{52}-1$ getallen tussen.
\item We weten dat $7$ en $9$ respectievelijk als volgt voorgesteld worden.
\[7 = .11100\ldots00 \cdot 2^3\]
\[9 = .10010\ldots00 \cdot 2^4\]
Hier zitten dus $2^{50}+2^{49}-1$ getallen tussen.
\end{itemize}

\end{document}