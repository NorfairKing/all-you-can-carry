\documentclass[samenvatting.tex]{subfiles}
\begin{document}

\chapter{Foutenanalyse}
\begin{defi}
Klassieke voorstelling.
\[
x = \sum_{i=m}^nc_{i}r^{i}\ \text{ }\ -\infty \le m \le 0 \le n \le \infty
\]
Grondtal (radix) $r$, Getallen voor de komma $n$, Getallen na de komma $|m|$.
\end{defi}
\begin{defi}
Bewegende kommavoorstelling.
\[
x = yb^e
\]
Mantisse $m$, Basis $b$, Exponent $e$. ($b^e$ is de schaalfactor)
\end{defi}
\begin{defi}
Exacte waarde van $x$: $x$.
\[
x = \sum_{i=m}^nc_{i}b^{i}
\]
Mantisse $m$, Basis $b$, Exponent $e$.
\end{defi}
\begin{defi}
Benadering voor $x$: $\overline{x}$.
\[
x = \sum_{i=m}^n\overline{c_{i}}b^{i}
\]
Mantisse $m$, Basis $b$, Exponent $e$.
\end{defi}
\begin{defi}
Absolute fout $\Delta x$.
\[
\Delta x = \overline{x} - x
\]
\end{defi}
\begin{defi}
Relatieve fout $\delta x$.
\[
\delta x = \frac{\overline{x} - x}{x}
\]
\end{defi}



\begin{defi}
Juist cijfer $c_i$:
\[
|\overline{x}-x| \le \frac{1}{2}b^i
\]
\end{defi}
\begin{defi}
Verband tussen absolute fout en aantal juiste cijfers na de komma.
\[
\frac{1}{2}b^{-p-1} < |\overline{x}-x| \le \frac{1}{2}b^{-p}
\]
Aantal juiste cijfers na de komma $p$.
\end{defi}
\begin{defi}
Verband tussen relatieve fout en aantal juiste cijfers na de komma.
\[
\frac{1}{2}b^{j-k-1} < \frac{|\overline{x}-x|}{\overline{x}} \le \frac{1}{2}b^{j-k+1}
\ \text{ of }\ 
\frac{1}{2}b^{-q-1} < \frac{|\overline{x}-x|}{\overline{x}} \le \frac{1}{2}b^{-q+1}
\]
Positie van het eerste beduidende cijfer $k$, Positie van het laatste beduidende cijfer $j+1$, Aantal juiste beduidende cijfers $q=k-j$.
\end{defi}
\begin{defi}
Machineprecisie $\epsilon_{mach}$.
\[
\epsilon_{mach} = \frac{1}{2}b^{1-p}
\]
\end{defi}
\begin{defi}
Het getallenbereik $O_{real}$: Alle $y$ zodat:
\[
y = mb^r \text{ is voorstelbaar in de machine}
\]
Mantisse $m$, basis $b$ en exponent $e$.
\end{defi}
\begin{defi}
Elementaire bewerking $\tau$.
\[
fl(x\ \tau\ y) = (x\ \tau\ y)(1 + \eta)
\]
$(|\eta| \le \epsilon_{mach}$ en $x,y \in O_{real})$
\end{defi}
\begin{defi}
Absolute fout op een som.
\[
\Delta\left(\sum_{i=1}^nx_i\right) = \sum_{i=1}^n\Delta x_i
\]
Bovengrens:
\[
\left|\Delta\left(\sum_{i=1}^nx_i\right)\right|_{max} = n\epsilon_{+} \le n \epsilon
\]
Relatieve fout op een som.
\[
\delta s = \frac{\Delta x + \Delta y}{x+y}
\]
Zie p 27 voor meer uitleg.
\end{defi}
\begin{defi}
Zij $\overline{x} = x(1+\delta x)$ en $\overline{y} = x(1+\delta y)$.\\
Absolute fout op een vermenigvuldiging.
\[
\Delta p = y\Delta x + x\Delta y
\]
Relatieve fout op een vermenigvuldiging. 
\[
\Delta xy = xy(\delta x + \delta y + \delta x \delta y) \approx \delta x + \delta y
\]
Bovengrens:
\[
|\delta(xy)| \le 2\epsilon_{\cdot} \text{ en } \left|\delta\left(\frac{x}{y}\right)\right| \le 2\epsilon_{\cdot}
\]
\end{defi}
\begin{defi}
Absolute fout op differentieerbare functies.
\[
\Delta f(x) = f(\overline{x}) - f(x) = \Delta x f'(x') \approx f'(\overline{x})\Delta x\text{ met } x' \text{ tussen } x \text{ en }\overline{x}
\]
Bovengrens:
\[
|\Delta f(x)|_{max} \approx |\Delta x|_{max} \underset{t}{max}|f'(t)|
\]
\[
|\Delta f(x_1,...,x_n)|_{max} \approx \sum_{i=1}^n|\Delta x_i|_{max} \underset{t_1,...,t_n}{max}|f_i'(t_1,...,t_n)|
\]
Zie p 30 voor voorbeelden.
\end{defi}
\begin{defi}
Norm $\Vert \cdot \Vert$.\\
De norm van een mathematisch object geeft de `grootte` ervan aan.
\end{defi}
\begin{defi} Gegevens en resultaten.\\
Exact gegeven $g$.\\
Gewijzigd gegeven $\overline{g}$.\\
Absoluut verschil in gegevens $\Delta g$.
\[
\Delta g = \overline{g} - g
\]
Relatief verschil in gegevens $\delta g$.
\[
\delta g
= \frac{\overline{g} - g}{\Vert g \Vert}
= \frac{\Delta g}{\Vert g \Vert}
\]
Exact resultaat $r$.
\[
r = F(g)
\]
Berekend resultaat $\overline{r}$
\[
\overline{r} = F(\overline{g})
\]
Absoluut verschil in resultaat $\Delta r$
\[
\Delta r
= F(\overline{g}) - r
= \overline{r} - r
\]
Relatief verschil in resultaat $\delta r$
\[
\delta r
= \frac{F(\overline{g}) - r}{\Vert r \Vert}
= \frac{\Delta r}{\Vert r \Vert}
\]
\end{defi}
\begin{defi}
Conditie van een probleem.\\
Absoluut Conditiegetal $k_A$.
\[
k_A =
\lim_{\epsilon \rightarrow 0}\sup_{\Vert \Delta g\Vert \le \epsilon} \frac{\Vert \Delta r\Vert}{\Vert \Delta g \Vert}
\]
Relatief Conditiegetal $k_R$
\[
k_R =
\lim_{\epsilon \rightarrow 0}\sup_{\Vert \Delta g\Vert \le \epsilon} \frac{\Vert \delta r\Vert}{\Vert \delta g \Vert}
\]
Het conditiegetal van een probleem geeft aan hoeveel een fout op de gegevens wordt opgeblazen in de resultaten.
\end{defi}
\noindent Voorbeeld: Als het relatieve conditiegetal van een probleem $F$ $a$ is, dan worden fouten op de gegevens met een factor $a$ opgeblazen in het slechtste geval.

\begin{defi} Benaderingen.\\
\begin{itemize}
\item exacte gegevens: $g$

\item exact resultaat: $r$

\item exact verband: $F$

\item Het exact verband tussen het exact resultaat $r$ van exact $F$ toegepast op exacte gegevens $g$.
\[
r = F(g)
\]

\item Inexacte gegevens: $\overline{g} = g + \Delta g = g(1+\delta g)$

\item De exacte berekening van het resultaat $\overline{r}$(, met ge\"induceerde fout $\Delta r$, $\delta r$,) gebaseerd op een gewijzigd gegeven $\overline{g}$.
\[
r + \Delta r = F(g + \Delta g) \rightarrow r(1+\delta r) = F(g(1+\delta g)) \rightarrow \overline{r} = F(\overline{g})
\]

\item Benaderd resultaat bij eindige discretisatie $\widetilde{r}$

\item Eindige discretisatie van $F$, exact berekend: $\widetilde{F}$. Discretisatiefout $err_{dis} = \widetilde{F} - F$.
\[
\widetilde{F} = F + err_{dis}
\]
\[
\widetilde{r} = \widetilde{F}(\overline{g})
\]
$\widetilde{F}$ verschilt dus enkel van $F$ in de discretisatie fouten.

\item Resultaat bij een echte berekening $\overline{\widetilde{r}}$ bij een berekening gebaseerd op inexacte gegevens $\overline{g}$.

\item 
Implementatie van $\widetilde{F}$ in eindige precisie.
\[
\overline{\widetilde{r}} = \overline{F}(\overline{g})
\]
De waarde van de gegevens als $\overline{\widetilde{r}}$ een exacte berekening zou zijn: $\overline{\overline{g}}$.
\[
\overline{F}(\overline{g}) = \overline{\widetilde{r}} = \widetilde{F}(\overline{\overline{g}})
\]

\item Het resultaat van het exact verband $F$ toegepast op inexacte gegevens $\overline{g}$: $\overline{\overline{r}}$

\end{itemize}
\end{defi}

\begin{defi}
Stabiliteit van een methode\\
\begin{itemize}
\item Voorwaartse/Sterke Stabiliteit.\\
Absolute voorwaartse stabiliteit:
De absolute grootte van het verschil tussen het berekende resultaat van de methode toegepast op een gewijzigd resultaat $\overline{g}$ in eindige precisie: $\overline{F}(\overline{g})$ en het exacte resultaat van de methode toegepast op een gewijzigd resultaat $\overline{g}$: $\widetilde{F}(\overline{g})$.
\[
\Vert \overline{F}(\overline{g}) - \widetilde{F}(\overline{g}) \Vert
=
\Vert \overline{\widetilde{r}}-\widetilde{r} \Vert
\]
Relatieve voorwaartse stabiliteit:
De relatieve grootte van het verschil tussen het berekende resultaat van de methode toegepast op een gewijzigd resultaat $\overline{g}$ in eindige precisie: $\overline{F}(\overline{g})$ en het exacte resultaat van de methode toegepast op een gewijzigd resultaat $\overline{g}$: $\widetilde{F}(\overline{g})$.
\[
\frac{\Vert \overline{F}(\overline{g}) - \widetilde{F}(\overline{g}) \Vert}{\Vert \widetilde{F}(\overline{g}) \Vert}
=
\frac{\Vert \overline{\widetilde{r}}-\widetilde{r} \Vert}{\Vert \widetilde{r} \Vert}
\]
Kleine waarde(n) $\Rightarrow$ Voorwaarts stabiele methode $\widetilde{F}$.

\item Achterwaartse Stabiliteit.\\
Absolute achterwaartse stabiliteit: De grootte van het verschil tussen de gegevens zoals ze zouden zijn als $\overline{\widetilde{r}}$ een exact resultaat was $\overline{\overline{g}}$ en de echte (gewijzigde) gegevens $\overline{g}$. 
\[
\Vert \overline{\overline{g}} - \overline{g} \Vert
\]
Relatieve achterwaartse stabiliteit: De relatieve grootte van het verschil tussen de gegevens zoals ze zouden zijn als $\overline{\widetilde{r}}$ een exact resultaat was $\overline{\overline{g}}$ en de echte (gewijzigde) gegevens $\overline{g}$. 
\[
\frac{\Vert \overline{\overline{g}} - \overline{g} \Vert}{\Vert \overline{g} \Vert}
\]
Relatieve waarde is $O(\epsilon_{mach})$ $\Rightarrow$ Achterwaarts stabiele methode.

\item Zwakke Stabiliteit.\\
Het de grootte van het verschil tussen $\overline{\overline{r}}$ en $\widetilde{r}$. ten opzichte van de grootte van het verschil tussen $\overline{\widetilde{r}}$ en $\widetilde{r}$: $S$.
\[
S =
\frac
{\Vert F(\overline{g}) - \widetilde{F}(\overline{g}) \Vert}
{\Vert \overline{F}(\overline{g}) - \widetilde{F}(\overline{g})\Vert}
=
\frac
{\Vert \overline{\overline{r}} - \widetilde{r} \Vert}
{\Vert \overline{\widetilde{r}} - \widetilde{r}\Vert}
\]
$S \approx 1$ $\Rightarrow$ zwak stabiele methode.
\end{itemize}
Merk op: dit zou niet zo specifiek gevraagd mogen worden, de tekst is niet duidelijk genoeg.
\end{defi}

\end{document}
