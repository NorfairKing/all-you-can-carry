\documentclass[samenvatting.tex]{subfiles}
\begin{document}

\chapter{Numerieke Integratie}
\section{Het principe}
Gegeven een functie in tabelvorm $\left\{ (x_i,f(x_i))\right\}_{i=0}^{n}$ waarvan we een integraal $\int_{a}^{b}f(x)$ willen berekenen. Zoek de interpolerende veelterm $y_n(x)$ en integreer deze over $[a,b]$ om een benadering te vinden voor de integraal van $f$.

\section{Conditie en stabiliteit}
Integratie is een relatief goed geconditioneerd probleem.
De stabiliteit van deze methode relatief goed, zeker als we ervoor zorgen dat we nergens verschillen berekenen (en dat is mogelijk).
De stabiliteit wordt bepaald door de discretisatiefout $F_n$ en de afrondingsfouten bij het evalueren van $\Sigma H_if(x_i)$

\section{Newton-Cotes}
Niet echt nuttig voor grote $n$.

\end{document}
