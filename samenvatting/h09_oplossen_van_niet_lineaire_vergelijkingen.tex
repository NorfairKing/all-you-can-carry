\documentclass[samenvatting.tex]{subfiles}
\begin{document}

\chapter{Oplossen van niet-lineaire vergelijkingen}

\section{Bissectie-methode}
\subsection{Voorwaarden}
\begin{itemize}
\item $f$ is continu over het gebied dat ons interesseert (noem het $[a,b]$).
\item $f(a)f(b) <0$, dit betekent dat er een nulpunt ligt in $[a,b]$
\end{itemize}
\subsection{methode}
$x^{(0)}=a$, $x^{(1)}=b$.
Deel het interval in twee en neem het linkse/rechtste deelinterval als nieuw interval afhankelijk van waar het nulpunt ligt.

\section{Secant-methode}
Kies $x^{(0)}$ en $x^{(1)}$.
\[
x^{(n+1)} = F(x^{(n)}, x^{(n-1)}) \text{ met } F(x,y) = y-\frac{(y-x)f(x)}{f(y)-f(x)}
\]
\begin{itemize}
\item Convergeert niet altijd.
\item Lineaire interpolatie
\end{itemize}

\section{Regula-falsi}
Kies $x^{(0)}$ en $x^{(1)}$.
\[
x^{(n+1)} = F(x^{(n)}, x^{(n-1)}) \text{ met } F(x,y) = y-\frac{(y-x)f(x)}{f(y)-f(x)}
\]
Kies nu uit $(x^{(0)},x^{(2)})$ en $(x^{(1)},x^{(2)})$ het interval waar de wortel in zit en ga verder.
\begin{itemize}
\item Combineert de eerste twee methodes, convergeert altijd.
\item Lineaire interpolatie
\end{itemize}

\section{Newton-Raphson}
\[
x^{(n+1}) = f(x^{(n)}+\frac{f(x^{(n)})}{f'(x^{(n)})} 
\]
\begin{itemize}
\item Substitutiemethode
\item Lineaire hermite interpolatie
\end{itemize}

\section{Whittaker}
Vervang in Newton-Raphson $f'(x^{(n)})$ door een (andere) benadering voor de afgeleide.
\begin{itemize}
\item Substitutiemethode
\item Lineaire hermite interpolatie
\item Sneller dan Newton-Raphson
\end{itemize}

\section{Muller}
Interpoleer quadratisch, begin met drie startwaarden.
\[ 
F(x,y,z) = f(z) + f[z,y](x-z)+ f[z,y,x](x-z)(x-y)
\]
Kies het nulpunt dat het dichtst bij $z$ ligt als nieuwe iteratiewaarde.

\begin{itemize}
\item Quadratische interpolatie
\end{itemize}

\section{Substitutiemethodes}
Substitutiemethodes zijn op zoek naar een vast punt.

\section{Convergentiesnelheid}
\subsection{Convergentiefactor}
\[
\epsilon^{(k)} = x^{(k)}-x^{*}
\]
\[
\rho^{(k)} = \frac{e^{(k)}}{e^{(k-1)}}
\]
Convergentiefactor: $\rho$
\[
\rho = \lim_{k\rightarrow\infty}\rho^{(k)}
\]
Het proces convergeert als $\rho < 1$.
\subsection{Orde van convergentie}
$p$ zodat het volgende geldt:
\[
e^{(k+1)} = O((e^{(k)})^p)
\]
Een proces waarbij $p = 1$ geldt, noemt men lineair.
Met een $p$ ussen $1$ en $2$ noemt men het superlineair.
Een proces waarbij $p=2$ geldt, noemt men kwadratisch.


\end{document}
