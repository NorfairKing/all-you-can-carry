\documentclass[samenvatting.tex]{subfiles}
\begin{document}

\chapter{Stelsels Lineaire Vergelijkingen}

\section{Algoritmes voor oplossen}
\subsection{Achterwaartse substitutie $O(n^2)$}
\subsubsection*{Input}
Een bovendriehoekig stelsel $A \in \mathbb{R}^{n\times n}$ en een vector $b \in \mathbb{B}^{n}$.
\[
A =
\begin{pmatrix}
a_{11} & a_{12} & \cdots & a_{1n}\\
0 & a_{22} & \cdots & a_{2n}\\
\vdots & \vdots & \ddots & \vdots\\
0 & 0 & \cdots & a_{nn}
\end{pmatrix}
\ { }\ 
b =
\begin{pmatrix}
b_{1}\\b_{2}\\\vdots\\b_{n}
\end{pmatrix}
\]
\subsubsection*{Output}
Een vector $X$, als oplossing voor het stelsel $AX=b$.

\subsection{Voorwaartse substitutie $O(n^2)$}
\subsubsection*{Input}
Een beneden driehoekig stelsel $A \in \mathbb{R}^{n\times n}$ en een vector $b \in \mathbb{B}^{n}$.
\[
A =
\begin{pmatrix}
a_{11} & 0 & \cdots & 0\\
a_{21} & a_{22} & \cdots & 0\\
\vdots & \vdots & \ddots & \vdots\\
a_{n1} & a_{n2} & \cdots & a_{nn}
\end{pmatrix}
\ { }\ 
b =
\begin{pmatrix}
b_{1}\\b_{2}\\\vdots\\b_{n}
\end{pmatrix}
\]
\subsubsection*{Output}
Een vector $X$, als oplossing voor het stelsel $AX=b$.

\subsection{Eliminatie $O(n^3)$}
\subsubsection*{Input}
Een matrix $A \in \mathbb{R}^{n\times m}$ met $m \ge n$.
\subsubsection*{Output}
Een boventrapeziummatrix $C$.
Merk op dat $c_{ij}$ de volgende waarde heeft.
\[
c_{ij} = c_{ji} - \sum_{l=i+1}^m c_{jl}\frac{c_{ji}}{c_{ii}}
\]

\subsection{Eenvoudige Gauseliminatie $O(n^3)$}
\subsubsection*{Input}
Een matrix $A \in \mathbb{R}^{n\times m}$ met $m>n$ en een vector $b \in \mathbb{B}^{n}$.\\
We beschouwen de matrix $C = [A|b]$.
\subsubsection*{Output}
Een vector $X$, als oplossing voor het stelsel $AX=b$.
\subsubsection*{Procedure}
Voer Eliminatie uit op de uitgebreide matrix $C$. Voer daarna voorwaartse substitutie uit op het resultaat om de vector $X$ te bekomen.

\subsection{Gaus-Jordan $O(n^3)$}
\subsubsection*{Input}
Een matrix $A \in \mathbb{R}^{n\times m}$ met $m>n$ en een vector $b \in \mathbb{B}^{n}$.\\
We beschouwen de matrix $C = [A|b]$.
\subsubsection*{Output}
Een vector $X$, als oplossing voor het stelsel $AX=b$.
\subsubsection*{Procedure}
Voer de eliminatie en substitutie tegelijk uit. Ga zo kolom per kolom af tot de originele $A$ in een (pseudo-)eenheidsmatrix is veranderd.

\subsection{Gauss met pivotering $O(n^3)$}
\subsubsection*{Input}
Een matrix $A \in \mathbb{R}^{n\times m}$ met $m>n$ en een vector $b \in \mathbb{B}^{n}$.\\
We beschouwen de matrix $C = [A|b]$.
\subsubsection*{Output}
Een vector $X$, als oplossing voor het stelsel $AX=b$.
\subsubsection*{Procedure}
Voer de methode van Gauss uit, maar verwissel rijen zodat het spilelement niet nul is.

\subsection{Determinant berekenen $O(n^3)$}
\subsubsection*{Input}
Een vierkante matrix $A\in\mathbb{R}^{n\times n}$.
\subsubsection*{Output}
De determinant van $A$.
\subsubsection*{Procedure}
Voer de eliminatiemethode uit op $A$, en bereken het product van de spilelement.

\subsection{Gauseliminatie met pivotering en factorizatie $O(n^3)$}
\subsubsection*{Input}
Een matrix $C = [A|B|l]$ met $ l = \left(1\ 2\ \cdots\ n\right)$:
\[
C = 
\left(
\begin{array}{ccc|c|c}
 & & & &1\\
 &A& &B&\vdots\\
 & & & &n\\ 
\end{array}
\right)
\]

\subsubsection*{Output}
Een benedendriehoeksmatrix $L$, een boventrapeziummatrix $R$ en een vector $Y$ zodat $AX=B$ equivalent is met $RX = Y$. Ook nog een matrix $P$ zodat $PA = LR$ en $PB = LY$.

\subsection{Gauseliminatie met optimale pivotering $O(n^4)$}
Als spilelement kiezen we elke keer het grootste element op die kolom en verwisselen dan die rijen. Zo wordt de stabiliteit van dit algoritme verhoogt.

\subsection{Methode van Crout}
\subsubsection*{Input}
Een matrix $A \in \mathbb{R}^{n\times m}$ met $m>n$ en een vector $b \in \mathbb{B}^{n}$.\\

\subsubsection*{Output}

\subsubsection*{Procedure}
Bereken een LR ontbinding door middel van eliminatie.
\[
C = LR
\]
\[
l_{ij} = c_{ij} - \sum_{k=1}^{j-1}l_{ik}r_{kj}
\]
\[
r_{ij} = \frac{c_{ij}-\sum_{k=1}^{i-1}l_{ik}r_{kj}}{l_{ii}} 
\]
Los daarna $RX = Y$ op via achterwaartse substitutie.

\subsection*{Samenvatting}
\begin{center}
\begin{tabular}{|c|c|}
\hline
Methode & Complexiteit\\
\hline
Voorwaartse substitutie & $O(n^2)$\\
Achterwaartse substitutie & $O(n^2)$\\
Eliminatie & $O(n^3)$\\
Eenvoudige Gauseliminatie & $O(n^3)$\\
Gaus-Jordan & $O(n^3)$.\\
Methode van Cramer & $O(n^4)$\\
Gauseliminatie met pivotering & $O(n^3)$\\
Determinant berekenen & $O(n^3)$\\
Gauseliminatie met pivotering en factorizatie & $O(n^3)$\\
Oplossen van meerdere stelsels met&$O(n)$\\ dezelfde co\"efficienten na factorisatie & Per vector\\
Gauseliminatie met optimale pivotering & $O(n^4)$\\
\hline
\end{tabular}
\end{center}

\section{Conditie van een stelsel lineaire vergelijkingen}
Norm van een matrix $A$: $\Vert A \Vert$
\[
\Vert A \Vert = \underset{X \neq 0}{max}\frac{\Vert AX \Vert}{\Vert X \Vert}
\]
Het conditiegetal van een matrix $A$: $\kappa(A)$
\[
\kappa(A) = \Vert A \Vert \Vert A^{-1} \Vert
\]
Residu van een stelsel lineaire vergelijkingen $AX=B$: $R$
\[
R = AX - B
\]

\subsubsection*{Onthoudt:}
\[
\Vert A\Vert\Vert A^{-1}\Vert
\]
Bovenstaande uitdrukking is het conditiegetal van een matrix $A$.

\section{Nullen Cre\"eren in een matrix}
\subsection{Givens Transformatie}
\[
G = 
\begin{pmatrix}
c & s\\
s & -c\\
\end{pmatrix}
=
\begin{pmatrix}
\cos\theta & \sin\theta\\
\sin\theta & -\cos\theta\\
\end{pmatrix}
\]
Zoek $y$ en $y$ zodat:
\[
\begin{pmatrix}
c & s\\
s & -c\\
\end{pmatrix}
\begin{pmatrix}
x\\y\\
\end{pmatrix}
=
\begin{pmatrix}
z\\0
\end{pmatrix}
\]
Oplossing:
\[
\left\{
\begin{array}{c c}
c &= \frac{x}{\pm \sqrt{x^2+y^2}}\\
s &= \frac{y}{\pm \sqrt{x^2+y^2}}\\
\end{array}
\right.
\]

\subsection{Householder Transformatie}
\[
P =\mathbb{I} - \frac{2vv^T}{v^Tv}
\]

\subsection{QR-factorisatie van een matrix}
Gegeven een matrix $A\in \mathbb{R}^{m\times n}$ zoek $Q$ en $R$ zodat.
\[
A = Q \cdot R \text{ en } Q^{T}Q = I
\]
met givens en householder transformaties.

\end{document}
