\documentclass[samenvatting.tex]{subfiles}
\begin{document}

\chapter{Numerieke differentiatie}
\section{Het principe}
Gegeven een functie in tabelvorm $\left\{ (x_i,f(x_i))\right\}_{i=0}^{n}$ waarvan we de afgeleide $f'(x)$ willen berekenen. Zoek de interpolerende veelterm $y_n(x)$ en leidt deze af om een benadering te vinden voor de afgeleide van $f$.

\section{Conditie en stabiliteit}
Differentiatie is een slecht geconditioneerd probleem. De conditie kan willekeurig slecht zijn.\\\\
De stabiliteit van deze methode is zeer slecht omdat er een verschrikkelijk instabiele stap in voorkomt.
De differentiatiefout is het grootst bij de interpolatiepunten en het kleinst in het midden van de intervallen.

\end{document}
