\documentclass[samenvatting.tex]{subfiles}
\begin{document}

\chapter{Veelterminterpolatie}
\begin{defi}
Gegeven $n+1$ punten $(x_i,f(x_i))$ is de veelterm van graad $n$: $y_n(x)$ de interpolerende veelterm.
\[
y_n(x) = a_0 + a_1x+ \cdots + a_nx^n = \sum_{i=0}^na_ix^i
\]
Voor $y_n$ geldt:
\[
y_n(x_i) = f_i
\]
\end{defi}
Het vinden van de waarde van de interpolerende veelterm kan via de co\"efficienten en een evaluatie of rechtstreeks.

\section{Interpolatie volgens Gauss}
Los het Vandermonde-stelsel op dat de interpolatievoorwaarden bepalen.
\[
\begin{pmatrix}
1 & x_0 & x_0^2 & \cdots & x_{0}^{n}\\
1 & x_1 & x_2^2 & \cdots & x_{1}^{n}\\
\vdots & \vdots & \vdots & \ddots &\vdots\\
1 & x_n & x_n^2 & \cdots & x_{n}^{n}
\end{pmatrix}
\begin{pmatrix}
a_0\\a_1\\\vdots\\a_n
\end{pmatrix}
=
\begin{pmatrix}
f(x_0)\\f(x_1)\\\vdots\\f(x_n)
\end{pmatrix}
\]
Dit kan in $O(n^3)$ tijd het co\"effici\"entenprobleem oplossen. We zullen deze methode enkel gebruiken om te tonen dat de interpolerende veelterm uniek is.

\section{Conditie en stabiliteit}
De conditie van beide problemen kan willekeurig slecht zijn. Wanneer het proces convergeert(, dus wanneer de interpolatiefout naar nul gaat, )zal de afrondingsfout stijgen. We zoeken dus een optimale graad $n$ voor interpolatie zodat de totale fout minimaal is.\\\\
We zien dat de interpolatiefout kleiner is in het middel van het interpolatieinterval en groter aan de uiteinden.

\section{Interpolatie volgens Lagrange (co\"efficienten)}
\[
y_n(x)
= l_0(x)f(x_0) + l_1(x)f_1(x_1) + \cdots + l_n(x)f(x_n)
= \sum_{i=0}^{n}l_i(x)f(x_i)
\]
Hierbij is $l_0$ de lagrange veelterm van graad $n$.
\[
l_{i} = \prod_{j=0,j\neq i}^{n}\frac{x-x_j}{x_i-x_j} = \frac{\pi(x)}{\pi'(x_i)(x-x_i)}
\]
Deze veelterm kan berekend worden in $O(n^2)$ tijd, daarna kan ze ge\"evalueerd worden in $O(n)$ tijd.
Deze vorm van interpolatie is geschikt als oplossing voor het co\"efficientenprobleem, maar niet voor het waardeprobleem.

\section{Interpolatie volgens Newton (beide)}
Om het waardenprobleem op te lossen willen we een methode die $y_{n+1}(x)$ makkelijk kan berekenen uit $y_n(x)$.
\[
y_n(x) = \sum_{i=0}^{n} f[x_0,x_1,...,x_i] \prod_{j=0}^{i-1}(x-x_j)
\]
Het waardeprobleem kan hiermee opgelost worden in $O(n)$ tijd.

\subsection{Gedeelde differenties}
\begin{defi}
Een gedeelde differentie is een waarde die de verandering van een functie voorstelt per afstand tussen opeenvolgende interpolatiepunten.
\[
f[x_i] = f(x_i)
\]
\[
f[x_i,x_{i+1},...,x_{j}] = \frac{f[x_{i+1},...,x_j] - f[x_i,...,x_{j-1}]}{x_j-x_i}
\]
\end{defi}
We kunnen deze gedeelde differenties berekenen door middel van de tabel der gedeelde differenties. De alternatieve tabel der gedeelde differenties dient om makkelijk de graad te kunnen verhogen.

\end{document}
