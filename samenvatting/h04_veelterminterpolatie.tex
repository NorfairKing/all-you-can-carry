\documentclass[samenvatting.tex]{subfiles}
\begin{document}

\chapter{Veelterminterpolatie}
\begin{defi}
Gegeven $n+1$ punten $(x_i,f(x_i)$ is de veelterm van graaf $n$: $y_n(x)$ de interpolerende veelterm.
\[
y_n(x) = a_0 + a_1x+ \cdots + a_nx^n = \sum_{i=0}^na_ix^i
\]
Voor $y_n$ geldt:
\[
y_n(x_i) = f_i
\]
\end{defi}
Het vinden van de waarde van de interpolerende veelterm kan via de co\"efficienten en een evaluatie of rechtstreeks.

\section{Interpolatie volgens Lagrange (co\"efficienten)}
\[
y_n(x)
= l_0(x)f(x_0) + l_1(x)f_1(x_1) + \cdots + l_n(x)f(x_n)
= \sum_{i=0}^{n}l_i(x)f(x_i)
\]
Hierbij is $l_0$ de lagrange veelterm van graad $n$.
\[
l_{i} = \prod_{j=0,j\neq i}^{n}\frac{x-x_j}{x_i-x_j} = \frac{\pi(x)}{\pi'(x_i)(x-x_i)}
\]
Deze veelterm kan ge\"evalueerd worden in $O(n^2)$ tijd.
Deze vorm van interpolatie is geschikt als oplossing voor het co\"efficientenprobleem, maar niet voor het waardeprobleem.
\section{Interpolatie volgens Newton (waarden)}
\[
y_n(x) = \sum_{i=0}^{n} f[x_0,x_1,...,x_i] \prod_{j=0}^{i-1}(x-x_j)
\]

\subsection{Gedeelde differenties}
\begin{defi}
Een gedeelde differentie is een waarde die de verandering van een functie voorstelt per afstand tussen opeenvolgende interpolatiepunten.
\[
f[x_i] = f(x_i)
\]
\[
f[x_i,x_{i+1},...,x_{j}] = \frac{f[x_{i+1},...,x_j] - f[x_i,...,x_{j-1}]}{x_j-x_i}
\]
\end{defi}

\end{document}
