\documentclass[examenvragen.tex]{subfiles}

\begin{document}

\section{Numerieke differentiatie volgens Lagrange}

\subsection{Opgave} 
Zij $f$ een functie die we interpoleren op het interval $]-h,h[$ met de volgende formule.
\[
f'(x) = \frac{1}{h^2}\left( \frac{2x-h}{2}f(-h) - 2xf(0) + \frac{2x+h}{2}f(h) \right) + D(x)
\]
Geef een uitdrukking voor $D(x)$ en een bovengrens. Waar is de differentiatiefout het grootst?

\subsection{Informatie}
\begin{itemize}
\item Boek pagina 131: Hoofdstuk 5 Numerieke differentiatie
\end{itemize}

\subsection{Antwoord}
Dit valt allemaal letterlijk in de cursus op te zoeken.
De differentiatiefout wordt gegeven door volgende formule.
\[
D_{n}(x_{i}) = \frac{\pi'(x_{i})}{(n+1)!}f^{(n+1)}(\xi(x_{i})) 
\]
We hebben echter te maken met equidistante interpolatiepunten.
\[
D_{n}(x_i) = (-1)^{n-i}\frac{i!(n-i)!}{(n+1)!}h^{n}f^{(n+1)}(\xi(x_i))
\]
Een bovengrens voor de interpolatiefout is dan de volgende.
\[
|D_{n}(x_i)| \le \frac{|\pi'(x_i)|}{(n+1)!}\max_{x\in]-h,h[}|f^{(n+1)}(x)
\]
De differentiatiefout is het grootst in de buurt van de interpolatiepunten.

\end{document}
