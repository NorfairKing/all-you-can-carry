\documentclass[examenvragen.tex]{subfiles}

\begin{document}

\section{Methode van de machten}

\subsection{Opgave}
Gegeven volgende matrix $A$.
\[
\begin{pmatrix}
a & b\\
c & d
\end{pmatrix}
\]
\begin{multicols}{4}
\begin{itemize}
\item $a=10^{-5}$
\item $b = 10^{-5}$
\item $c = 3-a$
\item $d = ab-\frac{2}{b}$
\end{itemize}
\end{multicols}
Wat is het condititiegetal van deze matrix? Als we de methode van de machten gebruiken, wat is dan de orde van convergentie, en de convergentiefactor? 

\subsection{Informatie}
\begin{itemize}
\item Boek pagina 67: 9.3 Het condititiegetal $\kappa(A)$
\item Boek pagina 292: 6 Slotbemerkingen en samenvatting puntje 5
\end{itemize}

\subsection{Antwoord}
\subsubsection{Condititegetal}
Het conditiegetal van een matrix $A$ ziet er als volgt uit.
\[
\kappa(A) = \Vert A\Vert\Vert A^{-1}\Vert
\]
Lees $\Vert A\Vert$ als `de norm van $A$'. We kiezen hiervoor de \'e\'ennorm. De \'e\'ennorm het maximum van de somaties van de absolute waarden van de elementen per kolom.
\[
\Vert A\Vert_{1} = \max_{1\le i\le n}\sum_{j=1}^{n}|a_{ij}|
\]
In dit geval ziet dat er als volgt uit.
\[
\Vert A\Vert_{1} = \max \{|a|+|c|, |b|+|d|\} = |b|+|d| = b-ab+\frac{2}{b} \approx 2\cdot 10^{5}
\]
Om de norm van de inverse matrix $\Vert A^{-1} \Vert$ van $A$ te berekenen moeten we natuurlijk eerst de inverse matrix $A^{-1}$ van $A$ berekenen.
\[
A^{-1} = 
\frac{1}{ad-bc}
\begin{pmatrix}
d & -b\\
-c &a
\end{pmatrix}
\]
\[
\Vert A^{-1}\Vert_{1} = \max \left\{\frac{|d|+|c|}{|ad-bc|},\frac{|b|+|a|}{|ad-bc|}\right\}
\]
\[
|ad-bc| = |10^{-15}-2- 3\cdot 10^{-5} + 10^{-10}| \approx 2
\]
\[
\Vert A^{-1}\Vert_{1} \approx 10^{5}
\]
Het conditiegetal van $A$ is van grootteorde $10^{10}$. Deze matrix $A$ is dus verschrikkelijk geconditioneerd. Hier valt niets aan te doe.
\subsubsection{Eigenwaarden}
We zetten $A$ in echelonvorm om de eigenwaarden af te lezen op de hoofddiagonaal.
\[
\begin{pmatrix}
a & b\\
c & d\\
\end{pmatrix}
\longrightarrow
\begin{pmatrix}
a & b\\
0 & \frac{da}{c}-b
\end{pmatrix}
\]
De eigenwaarden van $A$ zijn dus de volgende $\lambda_i$.
\[
\lambda_1 = a = 10^{-5}
\]
\[
\lambda_2 = \frac{da}{c}-b
= \frac{(10^{-5}10^{-5}-\frac{2}{10^{-5}})10^{-5}}{3-10^{-5}}-10^{-5}
= \frac{10^{-15}-2-3\cdot10^{-5}+10^{-10}}{3-10^{-5}} \approx -\frac{2}{3}
\]
De tweede eigenwaarde $\lambda_2$ is hier dominant. 
\subsubsection{Convergentie}
De methode van de machten zal convergeren naar de eigenvector $E_2$ die bij $\lambda_2$ hoort. De convergentiefactor $\rho$ ziet er dan als volgt uit.
\[
\frac{\lambda_1}{\lambda_2} \approx \frac{-3\cdot 10^{-5}}{2}
\]
De convergentieorde is \'e\'en.



\end{document}
