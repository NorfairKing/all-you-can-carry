\documentclass[examenvragen.tex]{subfiles}

\begin{document}

\section{Verschillende basis}

\subsection{Opgave}
Gegeven de volgende tweedegraads functie $f(x)$.
\[
f(x) = x^{2} + 2x + c \text{ met } c < 1
\]
Bespreek de conditie van de nulpunten van deze functie $f(x)$.

\subsection{Informatie}
\begin{itemize}
\item Boek pagina 238: 18 De conditite van een wortel
\item Boek pagina 239: Figuur 2.11 Conditie van het snijpunt $x^{*}$
\end{itemize}

\subsection{Antwoord}
Hoe dichter de afgeleide van de functie $f'(x)$ bij nul komt rond $x^{*}$, hoe slechter het nulpunt geconditioneerd is. Met andere woorden, hoe vlakker de functie bij het nulpunt, hoe slechter de conditie.
We onderzoeken hoe de afgeleide van de functie zich gedraagt bij de nulpunten.
Eerst berekenen we de nulpunten.
\[
x = \frac{-2 \pm \sqrt{4-4c}}{2}
\]
Vervolgens berekenen we de afgeleide van $f(x)$.
\[
f'(x) = 2x+2
\]
Tenslotte vullen we het nulpunt in in de afgeleide om te zien hoe die zich er gedraagt.
\[
f'(x^*) = 2 \frac{-2 \pm \sqrt{4-4c}}{2} + 2 = -2 \pm \sqrt{4-4c} + 2 = \pm 2\sqrt{1+c}
\]
Hier zien we dat de nulpunten slecht geconditioneerd is wanneer $c$ dicht bij $1$ ligt. Dit is natuurlijk precies wanneer het nulpunt tweevoudig is.

\end{document}
