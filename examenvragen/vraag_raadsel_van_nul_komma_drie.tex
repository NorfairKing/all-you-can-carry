\documentclass[examenvragen.tex]{subfiles}

\begin{document}

\section{Het raadsel van $0.3$}

\subsection{Opgave}
Gegeven volgend programma met de output. Leg in detail uit waarom het programma besluit dat $y$ niet gelijk is aan $0.3$.
\begin{lstlisting}
x = 0.1
y = 3 * 0.1
if y == 0.3
then print "y is gelijk aan 0.3"
else print "y is niet gelijk aan 0.3"
\end{lstlisting}
\begin{lstlisting}
x = 1.0000000000E-1
y = 3.0000000000E-1
y is niet gelijk aan 0.3
\end{lstlisting}

\subsection{Informatie}
\begin{itemize}
\item Boek pagina 13: Hoofdstuk 2 Foutenanalyse
\end{itemize}

\subsection{Antwoord}
Er worden in de eerste drie lijnen al drie fouten gemaakt in de machine.
In lijn $1$ moet de machine $0.1$ afronden omdat $0.1$ binair niet correct voorgesteld kan worden in eindige precisie.
\[
\overline{x} = x(1+\epsilon_1) \text{ met } |\epsilon_{1}| \le \epsilon_{mach}
\]
Waneer $x$ wordt afgedrukt wordt $x$ eerst terug omgezet naar het decimaal talstelsel. De fout op $x$ is te klein (kleiner dan de machineprecisie) om een verschil te maken in het decimaal talstelsel in de machine.
In de tweede lijn worden twee fouten gemaakt. Eerst wordt dezelfde fout gemaakt als in lijn $1$. $0.1$ wordt omgezet naar binair. Vervolgens wordt $0.1$ met $3$ vermenigvuldigd.
\[
\overline{y} = 0.3(1+\epsilon_1)(1+\epsilon_2) \text{ met } |\epsilon_{1}|,|\epsilon_{2}| \le \epsilon_{mach}
\]
In de derde lijn wordt $0.3$ omgezet naar binair. Op dit moment wordt er opnieuw een fout gemaakt.
\[
\overline{0.3} = 0.3(1+\epsilon_3) \text{ met }|\epsilon_{3}| \le \epsilon_{mach}
\]
Tenslotte wordt in de derde lijn $\overline{y}$ met $\overline{0.3}$ vergeleken.
\[
0.3(1+\epsilon_1)(1+\epsilon_2) - 0.3(1+\epsilon_3)
=
\epsilon_1+\epsilon_2-\epsilon_3 \neq 0
\]
Het verschil tussen $\overline{y}$ en $\overline{0.3}$ is groot genoeg om opgemerkt te worden door de machine. De machine beweert daarom dat $y$ niet gelijk is aan $0.3$.


\end{document}
