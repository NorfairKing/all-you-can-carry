\documentclass[examenvragen.tex]{subfiles}

\begin{document}

\section{Verschillende basis}
\subsection{Opgave}
Gegeven een tweedimensionaal lineair stelsel $Ax=b$.
\[
A = \begin{pmatrix}
\alpha + 1 & 1 \\
\alpha &1
\end{pmatrix}
\]
Bij welke waarden voor $\alpha$ convergeert de methode van Jacobi om nulpunten te vinden?
\subsection{Informatie}
\begin{itemize}
\item Boek pagina 271: 2 Convergentie van methodes
\end{itemize}
\subsection{Antwoord}
De methode van Jacobi en Gauss-Seidel convergeren als $A$ \emph{diagonaal dominant} is. Dit betekent dat volgende ongelijkheid geldt.
\[
\forall k \in \{1,\cdots,n\}: |a_{kk}| \ge \sum_{j=1,j\neq k}|a_{kj}|
\]
In mensentaal betekent dit dat een element op de diagonaal in absolute waarde groter is dan de elementen op dezelfde rij samen.
Voor elke rij krijgen we dus een ongelijkheid die moet gelden opdat de methode van Jacobi convergeert.
\[
\left\{
\begin{array}{c c}
|\alpha +1| > 1\\
1 > |\alpha|
\end{array}
\right.
\]
We werken dit stelsel voorzichtig uit. Er zitten namelijk absolute-waardetekens in die wel een last kunnen geven.
\[
|\alpha+1|>1 \Leftrightarrow \alpha + 1 > 1 \vee \alpha + 1 < -1 \Leftrightarrow \alpha > 0 \vee \alpha < -2
\]
\[
|\alpha|<1 \leftrightarrow \alpha < 1 \wedge \alpha > -1
\]
Nemen we deze samen, dan krijgen we het volgende interval voor $\alpha$.
\[
(\alpha > 0 \vee \alpha < -2) \wedge (\alpha < 1 \wedge \alpha > -1) \Leftrightarrow \alpha \in\ ]0,1[\ \cup\ ]-\infty,-2[
\]

\end{document}
