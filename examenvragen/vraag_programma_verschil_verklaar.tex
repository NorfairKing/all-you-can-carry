\documentclass[examenvragen.tex]{subfiles}

\begin{document}

\section{Programma verschil: Verklaar}
\subsection{Opgave}
Gegeven volgend programma met de output. Verklaar waarom het verschil plots niet meer gelijk is aan $0$. Waarvoor staat de nieuwe waarde van het verschil?
\begin{lstlisting}
som = 0.0
for i = 0.0:0.1:1.0
	verschil = i - som
	print(verschil)
	som = som + 0.1
end
\end{lstlisting}
\begin{lstlisting}
verschil = 0
verschil = 0
verschil = 0
verschil = 0
verschil = 0
verschil = 0
verschil = 1.1.. e-16
verschil = 1.1.. e-16
verschil = 1.1.. e-16
verschil = 1.1.. e-16
verschil = 1.1.. e-16
\end{lstlisting}
\subsection{Informatie}
\begin{itemize}
\item Boek pagina 22: 5.3 De machinenauwkeurigheid
\item Oefenzitting 2: Probleem 1 tot 4
\end{itemize}
\subsection{Antwoord}
We werken op de computer met een getallenvoorstelling in basis $b=2$ en een mantisse met een eindig aantal bits.
Het getal $0.1$ kan dus niet exact worden voorgesteld.
$0.1$ ziet er binair als volgt uit.
\begin{verbatim}
0.000110011001100110011001100110011001100110011001100110...
\end{verbatim}
In de machine moet dit getal dus worden afgerond of afgekapt, zodat er een fout gemaakt wordt. Die fout is klein maar bestaande.
Telkens wanneer er $0.1$ bij `som' wordt opgeteld stapelt diezelfde fout zich op. In iteratie $j$ ziet `som' er wiskundig gezien zo uit. 
\[
som = j(0.1)(1+\epsilon) = 0.1j + j\epsilon
\]
Hier is bovendien $\epsilon$, in absolute waarde, kleiner dan de machineprecisie $\epsilon_{mach}$. ($\epsilon$ blijkt negatief te zijn.)
Wanneer de waarde van `verschil' berekend wordt in iteratie $j$, blijven er eigenlijk enkel nog de opgestapelde fouten over.
\[
verschil = 0.1j - som = 0.1j - 0.1j -j\epsilon = -j\epsilon
\]
Wanneer `verschil' wordt afgedrukt wordt de waarde opnieuw omgezet naar het decimaal talstelsel. Als de opgestapelde fout dan nog kleiner is dan de machineprecisie $\epsilon_{mach}$ zal de uitvoer $0$ geven.
Vanaf een bepaalde iteratie zal de opgestapelde fout $j\epsilon$ echter groter worden dan de machineprecisie $\epsilon_{mach}$. Bij de omzetting naar het decimaal stelsel zal de uitvoer niet meer $0$ geven, maar het eerstvolgende getal dat voorgesteld kan worden in de machine.

Op het moment dat de opgestapelde fouten te groot worden, dus wanneer de uitvoer $1.1E-16$ geeft, is `som' (ongeveer) gelijk aan $0.5$. De afstand tussen $0.5$ en het volgende voorstelbaar getal in de machine is $eps(0.5) = 1.1E-16$. Dit is precies hetgeen de uitvoer geeft.


\end{document}
