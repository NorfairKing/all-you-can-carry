\documentclass[examenvragen.tex]{subfiles}

\begin{document}

\section{Kwadratuurformule met een zo hoog mogelijke nauwkeurigheidsgraad}
\subsection{Opgave}
Bepaal de gewichten $H_i$ van volgende kwadratuurformule zodat ze de hoogst mogelijke nauwkeurigheidsgraad heeft. Welke nauwkeurigheidsgraad is dat?
\[
\int_{a-h}^{a+h}f(x)dx = H_{-\frac{1}{2}}f\left(a-\frac{h}{2}\right) + H_0f(a) + H_\frac{1}{2}f\left(a+\frac{h}{2}\right)
\]

\subsection{Informatie}
\begin{itemize}
\item Tutorial: Kwadratuurformules
\item Boek pagina 139: 1 Interpolerende kwadratuurformules
\end{itemize}

\subsection{Antwoord}
Noem $d$ de minimale nauwkeurigheidsgraad. Noem de exacte integraal $I(f)$ en de kwadratuurformule $I_{k}(f)$.
Voor elke veelterm $p$ van een graad kleiner dan $d$ geldt dat de kwadratuurformule de integraal exact benadert.
Er bestaat minstens \'e\'en veelterm van graad $d+1$ die niet exact wordt benaderd door de kwadratuurformule. We kunnen $a$ in de oorsprong leggen zonder verlies van algemeenheid (door de grafiek mee te verschuiven). Er zijn drie gewichten in de kwadratuurformule, dus we leggen een nauwkeurigheidsgraad van $d=2$ op.
We lossen nu volgend stelsel op.
\[
\left\{
\begin{array}{ccl}
I_k(1) &= H_{-\frac{1}{2}} + H_0 + H_\frac{1}{2} &= 2h\\
I_k(x) &= -\frac{h}{2}H_{-\frac{1}{2}} + \frac{h}{2} H_\frac{1}{2} &= 0\\
I_k(x^2) &= \frac{h^2}{4}H_{-\frac{1}{2}} + \frac{h^2}{4} H_\frac{1}{2} &= \frac{2h^3}{3}\\
\end{array}
\right.
\]
Lossen we nu dit stelsel op, dan krijgen we voor de gewichten de volgende waarden.
\[
H_{-\frac{1}{2}} = H_\frac{1}{2} = \frac{4h}{3} \text{ en } H_0 = -\frac{2h}{3}
\]
Deze kwadratuurformule heeft dus minstens een nauwkeurigheidsgraad van $2$.
\[
\int_{a-h}^{a+h}f(x)dx =
\frac{4h}{3}f\left(a-\frac{h}{2}\right)- \frac{2h}{3}f(a) + \frac{4h}{3}f\left(a+\frac{h}{2}\right)
\]
Het blijkt zo te zijn dat de formule veeltermen van graad drie ook nog exact integreert, maar niet alle veeltermen van graad vier. De nauwkeurigheidsgraad van deze formule is dus $3$.


\end{document}
