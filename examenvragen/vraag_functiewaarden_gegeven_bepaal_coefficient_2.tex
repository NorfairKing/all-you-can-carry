\documentclass[examenvragen.tex]{subfiles}

\begin{document}

\section{Functiewaarden gegeven, bepaal coe\"efficient.}
\subsection{Opgave}
Gegeven de volgende informatie over een functie $f$.
\begin{multicols}{4}
\begin{itemize}
\item $x_0 = 0$
\item $x_1 = 1$
\item $f(x_0) = 2$
\item $f'(x_0) = 0$
\item $f'(x_1) = 0$
\item $f''(x_0) = 0$
\item $f''(x_1) = 0$
\item $f[x_0,x_0,x_1,x_1] = 1$
\end{itemize}
\end{multicols}
Bereken de waarde van $f(x_1)$.

\subsection{Informatie}
\begin{itemize}
\item Boek pagina 102: 9 Interpolatie volgens Newton
\end{itemize}

\subsection{Antwoord}
Let op: We weten niet of de functie $f(x)$ een veeltermfunctie is of niet!
De gedeelde differentie heeft natuurlijk iets te maken met een afgeleide.
\[
\lim_{x\rightarrow y}f[x,y] = \lim_{x\rightarrow y}\frac{f(y)-f(x)}{y-x} = f'(y)
\]
Op deze manier zullen we de gegeven waarden voor de afgeleiden kunnen gebruiken.
We zullen de gegeven derdegraads gedeelde differentie schrijven als een limiet, om bovenstaande formule te kunnen gebruiken.
\[
f[x_0,x_0,x_1,x_1] = \lim_{x_2\rightarrow x_0, x_3 \rightarrow x_1}[x_2,x_0,x_1,x_3]
\]
We zullen nu de gegeven gedeelde differentie van graad $3$ uitwerken volgens de definitie, om een vergelijking te bekomen met $f(x_1)$ als onbekende.
\[
\lim_{x_2\rightarrow x_0, x_3 \rightarrow x_1}
f[x_2,x_0,x_1,x_3]
=
\lim_{x_2\rightarrow x_0, x_3 \rightarrow x_1}
\frac
{f[x_2,x_0,x_1]-f[x_0,x_1,x_3]}
{x_0-x_1}
\]
\[
=
\frac
{\lim_{x_2\rightarrow x_0}f[x_2,x_0,x_1]-\lim_{x_3 \rightarrow x_1}f[x_0,x_1,x_3]}
{x_0-x_1}
=
\frac
{
\lim_{x_2\rightarrow x_0}
\frac{f[x_2,x_0]-f[x_0,x_1]}{x_2-x_1}
-
\lim_{x_3 \rightarrow x_1}\frac{f[x_0,x_1]-f[x_1,x_3]}{x_0-x_3}
}
{x_0-x_1}
\]
\[
=
\frac
{
\frac{\lim_{x_2\rightarrow x_0}f[x_2,x_0]-f[x_0,x_1]}{x_0-x_1}
-
\frac{f[x_0,x_1]-\lim_{x_3 \rightarrow x_1}f[x_1,x_3]}{x_0-x_1}
}
{x_0-x_1}
\]
\[
=
\frac
{
\frac{f'(x_0)-\frac{f(x_0)-f(x_1)}{x_0-x_1}}{x_0-x_1}
-
\frac{\frac{f(x_0)-f(x_1)}{x_0-x_1}-f'(x_1)}{x_0-x_1}
}
{x_0-x_1}
\]
Van deze uitdrukking weten we dat ze gelijk is aan $1$. (gegeven)
Vul nu in wat verder nog gegeven is en werk uit om $f(x_1)$ te bekomen.
\[
1
=
\frac
{
\frac{f'(x_0)-\frac{f(x_0)-f(x_1)}{x_0-x_1}}{x_0-x_1}
-
\frac{\frac{f(x_0)-f(x_1)}{x_0-x_1}-f'(x_1)}{x_0-x_1}
}
{x_0-x_1}
=
\frac
{
\frac{0-\frac{2-f(x_1)}{-1}}{-1}
-
\frac{\frac{2-f(x_1)}{-1}-0}{-1}
}
{-1}
\]
\[
1
=
2-f(x_1)+
2-f(x_1)
= 4-2f(x_1)
\]
\[
f(x_1) = \frac{3}{2}
\]


\end{document}
