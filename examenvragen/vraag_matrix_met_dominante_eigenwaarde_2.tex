\documentclass[examenvragen.tex]{subfiles}

\begin{document}

\section{Matrix met dominante eigenwaarde}
\subsection{Opgave}
Gegeven volgende matrix $A$ en startvector $x_0$.
\[
A =
\begin{pmatrix}
2 & 1 & -1\\
0 & 3 & -5\\
0 & 0 & -2
\end{pmatrix}
\text{ en }
x_0 = 
\begin{pmatrix}
-1\\1\\1
\end{pmatrix}
\]
Wat gebeurt als we de methode van `Von Mises' uitvoeren op de matrix $A$ met startvector $x_0$.

\subsection{Informatie}
\begin{itemize}
\item Boek pagina 287: Hoofdstuk 6 Het berekenen van eigenwaarden.
\item Boek pagina 290: Opmerking 5 (normalisatie)
\item Boek pagina 292: 6 Slotbemerkingen en samenvatting puntje 5 (convergentie- orde en factor)
\end{itemize}

\subsection{Antwoord}
De methode van `Von Mises' vermenigvuldigt de vector $x_0$ herhaalderlijk met $A$ tot er een dominante eigenvector overblijft.
De eigenwaarden van $A$ zijn af te lezen op de hoofddiagonaal: $3$, $2$ en $-2$. De methode zal convergeren naar de eigenvector die bij $3$ hoort.
\[
\left(
\begin{array}{ccc|c}
-1 & 1 & -1 & 0\\
0 & 0 & -5 & 0\\
0 & 0 & -5 & 0\\
\end{array}
\right)
\longrightarrow
\left(
\begin{array}{ccc|c}
-1 & 1 & 0 & 0\\
0 & 0 & 1 & 0\\
0 & 0 & 0 & 0\\
\end{array}
\right)
\longrightarrow
E = 
t
\begin{pmatrix}
1\\1\\0
\end{pmatrix}
\]
\subsubsection{Converentie snelheid}
De convergentiefactor ziet er als volgt uit.
\[
\frac{\lambda_2}{\lambda_1} = \pm \frac{2}{3}
\]
Omdat de convergentiefactor niet nul is, is de convergentie hoogstens lineair.
Er kan nog normalisatie gebruikt worden om overloop te voorkomen.
\end{document}
