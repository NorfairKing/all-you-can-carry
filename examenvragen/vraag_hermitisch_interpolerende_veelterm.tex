\documentclass[examenvragen.tex]{subfiles}

\begin{document}

\section{Verschillende basis}
\subsection{Opgave}
Gegeven volgende informatie over een functie $f$. Bepaalde Hermitisch interpolerende veelterm van graad $3$.
\begin{multicols}{4}
\begin{itemize}
\item $f(x_0)$
\item $f(x_1)$
\item $f'(x_0)$
\item $f'(x_1)$
\end{itemize}
\end{multicols}
\subsection{Informatie}
\begin{itemize}
\item Boek pagina 122: 11 Hermite interpolatie
\end{itemize}

\subsection{Antwoord}
We kunnen dit oplossen met de methode der onbepaalde co\"effici\"enten. We leggen vier interpolatievoorwaarden op voor de interpolerende veelterm van graad $3$.
\[
\left\{
\begin{array}{ccccccccc}
a_0 &+& a_1x_0 &+& a_2x_0^{2} &+& a_3x_{0}^{3} &=& f(x_0)\\
a_0 &+& a_1x_1 &+& a_2x_1^{2} &+& a_3x_{1}^{3} &=& f(x_1)\\
&& a_1 &+& 2a_2x_0 &+& 3a_3x_{0}^{2} &=& f'(x_0)\\
&& a_1 &+& 2a_2x_1 &+& 3a_3x_{1}^{2} &=& f'(x_1)\\
\end{array}
\right.
\]
\[
\begin{pmatrix}
1 & x_0 & x_0^2 & x_0^3\\
1 & x_0 & x_0^2 & x_0^3\\
0 & 1 & 2x_0 & 3x_0^2\\
0 & 1 & 2x_0 & 3x_0^2\\
\end{pmatrix}
\begin{pmatrix}
a_0\\a_1\\a_2\\a_3
\end{pmatrix}
=
\begin{pmatrix}
f(x_0)\\f(x_1)\\f'(x_0)\\f'(x_1)
\end{pmatrix}
\]
Dit stelsel kunnen we met de hand uitrekenen.

\end{document}
