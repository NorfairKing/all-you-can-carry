\documentclass[examenvragen.tex]{subfiles}

\begin{document}

\section{Functiewaarden gegeven, bepaal co\"eefficient.}
\subsection{Opgave}
Gegeven volgende informatie over een veelterm $p$ van graad $4$.
\[
p(x) = a_{0} + a_1x + a_2x^{2} +a_{3}x^{3}+a_{4}x^{4}
\]
\begin{multicols}{4}
\begin{itemize}
\item $p(0)=0$
\item $p(1)=9$
\item $p(2)=15$
\item $p(3)=18$
\end{itemize}
\end{multicols}
Alle gedeelde differenties van de vierde graad zijn bovendien $1$.
Bereken de waarde van de derde co\"efficie\"ent $a_3$.
\subsection{Informatie}
\begin{itemize}
\item Boek pagina 103: 9.3 Gedeelde differenties
\end{itemize}
\subsection{Antwoord}
Alle gedeelde differenties van de vierde graad zijn $1$. Dus ook de gedeelde differentie van de eerste vijf co\"efficienten.
\[
p[x_{0},x_{1},x_{2},x_{3},x_{4}] = 1
\]
De nulde differenties zijn ook gegeven voor vier verschillende waarden van $x$. ($x_0,\cdots,x_3$)
\begin{multicols}{4}
\begin{itemize}
\item $p[0]=0$
\item $p[1]=9$
\item $p[2]=15$
\item $p[3]=18$
\end{itemize}
\end{multicols}
\noindent We kunnen nu de gedeelde differenties van de eerste graad berekenen ...
\[
p[x,y] = \frac{p(y) - p(x)}{y-x}
\]
\begin{multicols}{3}
\begin{itemize}
\item $p[0,1] = 4$
\item $p[1,2] = 6$
\item $p[2,3] = 3$
\end{itemize}
\end{multicols}
\noindent ... alsook de gedeelde differenties van de tweede graad.
\begin{multicols}{2}
\begin{itemize}
\item $p[0,1,2] = 1$
\item $p[1,2,3] = -\frac{3}{2}$
\end{itemize}
\end{multicols}
Tenslotte berekenen we nog de gedeelde differenties van de derde graad.
\[
p[0,1,2,3] = -\frac{5}{6}
\]
We kunnen nu $p$ berekenen met interpolatie volgens newton.
\[
p(x) = 5+4(x-0)+1(x-0)(x-1)-\frac{5}{6}(x-0)(x-1)(x-2) + 1(x-0)(x-1)(x-2)(x-3)
\]
Werk dit uit om de derde co\"efficient $a_3$ af te lezen.
\[
a_3 = -6.833
\]

\end{document}
